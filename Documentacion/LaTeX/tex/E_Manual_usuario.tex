\apendice{Documentación de usuario}

\section{Introducción}
A continuación se describirán los requisitos mínimos a cumplir para que el usuario pueda entrar en la aplicación y usarla.

\section{Requisitos de usuarios}
 Al estar la aplicación desplegada en \href{https://gestor-tfg-2022.herokuapp.com/}{https://gestor-tfg-2022.herokuapp.com/} por lo que solamente hará falta disponer de Internet.
\section{Instalación}
Para utilizar la aplicación no será necesario instalar ningún componente en nuestro ordenador, a excepción de un navegador web.
\section{Manual del usuario}
A continuación se detallará el uso de la web, exclusivamente de las nuevas pantallas implantadas.
\subsection{Histórico profesorado}
En esta nueva pantalla implantada se expone información histórica sobre el profesorado.

En la parte superior de la pantalla tenemos una opción para actualizar la base datos utilizada. Se informa de la última modificación de la base de datos actual y se le advierte al usuario que esta actualización es un proceso lento que puede tardar al rededor de un minuto. Ver imagen \ref{fig:ActualizarDatos}.

\imagenflotante{ActualizarDatos}{Preguntar al usuario si quiere actualizar la base de datos}{0.9}

Si se ha optado por actualizar, se mostrará al final un aviso al usuario con el tiempo transcurrido durante la operación. %Ver imagen \ref{Notifiacion}

%\imagenflotante{Notificacion}{Notificación al usuario sobre la actualización}{0.9}

Posteriormente tenemos un pequeño apartado de información sobre el número de áreas, departamentos y profesores de la EPS. Ver imagen \ref{infor_profes}.

\imagenflotante{infor_profes}{Información sobre la EPS}{0.9}

A continuación debemos escoger las áreas, departamentos y profesores que deseemos visualizar en la gráfica. Una vez seleccionados \emph{clickaremos} en actualizar gráfica.
Si queremos seleccionar varios profesores debemos introducir primero uno y posteriormente el siguiente, para eliminarlos de la selección pulsaremos en la \emph{x} del nuevo botón que se añade tras indicar un tutor. Ver imagen \ref{fig:parametros}.

\imagenflotante{parametros}{Selección de parámetros}{0.9}

La gráfica muestra el número de TFGs por curso asignado a ese parámetro. %Ver imagen \ref{fig:grafica_anexo}

\imagenflotante{grafica_anexo}{Gráfica final tras seleccionar los parámetros}{0.9}

\subsection{Creación de informes}
En esta pantalla se le da al usuario la opción de añadir un área sobre el que se quiere hacer un informe, y el nombre que se le quiere dar. Ver imagen \ref{fig:pantalla_informe}.

\imagenflotante{grafica_anexo}{Gráfica final tras seleccionar los parámetros}{0.9}

Este informe contendrá el número total de TFGs dirigidos, codirigidos y el número de créditos asignados a cada uno de los profesores del área seleccionado en el último curso académico. Si se seleccionan varias áreas se crearán varias hojas en el documento \emph{excel} generado, con la información pertinente.

\subsection{Oferta de TFG}
\subsection{Aceptación TFG}

