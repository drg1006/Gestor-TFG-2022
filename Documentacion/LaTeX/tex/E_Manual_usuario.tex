\apendice{Documentación de usuario}

\section{Introducción}
A continuación se describirán los requisitos mínimos a cumplir para que el usuario pueda entrar en la aplicación y usarla.

\section{Requisitos de usuarios}
 Al estar la aplicación desplegada en \href{https://gestor-tfg-2022.herokuapp.com/}{https://gestor-tfg-2022.herokuapp.com/} por lo que solamente hará falta disponer de Internet.
\section{Instalación}
Para utilizar la aplicación no será necesario instalar ningún componente en nuestro ordenador, a excepción de un navegador web.
\section{Manual del usuario}
A continuación se detallará el uso de la web, exclusivamente de las nuevas pantallas implantadas.
\subsection{Histórico profesorado}
En esta nueva pantalla implantada se expone información histórica sobre el profesorado.

En la parte superior de la pantalla tenemos una opción para actualizar la base datos utilizada. Se informa de la última modificación de la base de datos actual y se le advierte al usuario que esta actualización es un proceso lento que puede tardar al rededor de un minuto. Ver imagen \ref{fig:ActualizarDatos}.

\imagenflotante{ActualizarDatos}{Preguntar al usuario si quiere actualizar la base de datos}{0.9}

Si se ha optado por actualizar, se mostrará al final un aviso al usuario con el tiempo transcurrido durante la operación. Ver imagen \ref{fig:Notifiacion}

\imagenflotante{Notificacion}{Notificación al usuario sobre la actualización}{0.9}

Posteriormente tenemos un pequeño apartado de información sobre el número de áreas, departamentos y profesores de la EPS. Ver imagen \ref{fig:infor_profes}.

\imagenflotante{infor_profes}{Información sobre la EPS}{0.9}

A continuación debemos escoger las áreas, departamentos y profesores que deseemos visualizar en la gráfica. Una vez seleccionados \emph{clickaremos} en actualizar gráfica.
Si queremos seleccionar varios profesores debemos introducir primero uno y posteriormente el siguiente, para eliminarlos de la selección pulsaremos en la \emph{x} del nuevo botón que se añade tras indicar un tutor. Ver imagen \ref{fig:parametros}.

\imagenflotante{parametros}{Selección de parámetros}{0.9}

La gráfica muestra el número de TFGs por curso asignado a ese parámetro. Ver imagen \ref{fig:graficaAnexo}

\imagenflotante{graficaAnexo}{Gráfica final tras seleccionar los parámetros}{0.9}

\subsection{Creación de informes}
En esta pantalla se le da al usuario la opción de añadir el número de alumnos matriculados en la asignatura \emph{Trabajos de Fin De Grado} y un área sobre el que se quiere hacer un informe, y el nombre que se le quiere dar. 
Ver imagen \ref{fig:pantalla_informes}.

\imagenflotante{pantalla_informes}{Gráfica final tras seleccionar los parámetros}{0.9}

Este informe contendrá el número total de TFGs dirigidos, codirigidos y el número de créditos asignados a cada uno de los profesores del área seleccionado en el último curso académico. Si se seleccionan varias áreas se crearán varias hojas en el documento \emph{excel} generado, con la información pertinente. 
Pestaña 1: ver imagen \ref{fig:informe_excel}
\imagenflotante{informe_excel}{Informe generado}{0.9}

Pestaña 2: ver imagen \ref{fig:informe_excel2}
\imagenflotante{informe_excel2}{Informe generado}{0.9}

\subsection{Oferta de TFG}

A esta nueva pantalla solo podrán acceder los usuarios de \emph{Moodle} que sean profesores, es decir se pedirá al usuario que se identifique con su \emph{email y contraseña}. Si no tienen los permisos necesarios se les negará el acceso.

En este nuevo menú se le pedirá al tutor que introduzca los parámetros necesarios para crear un nuevo TFG:

\begin{itemize}
	\item Titulo.
	\item Descripción.
	\item Tutor1, por defecto se indica el nombre del tutor que se ha \emph{loggeado}.
	\item Tutor2.
	\item Alumno1, por defecto se indica \emph{'Alumnos sin asignar'}.
	\item Alumno2.
	\item CursoAsignación, por defecto se indicará el curso actual.
\end{itemize}

Todos estos parámetros son modificables. Una vez se seleccionen, el usuario \emph{clickará} en \emph{Subir TFG} y se añadirá a la base de datos, en la pestaña de \emph{N2 Proyectos}, es decir, en los proyectos activos.
También se añadirá una nueva columna que indica si el TFG ha sido aceptado o si la petición está aún pendiente. En esta pestaña siempre se añadirán como \emph{pendientes} ya que en la siguiente será un administrador el que modifique su estado. Ver imagen \ref{fig:subirTFG}

\imagenflotante{subirTFG}{Pantalla para indicar la información del TFG}{0.9}

\subsection{Aceptación TFG}

A esta nueva pantalla solo podrán acceder los usuarios de \emph{Moodle} que sean administradores, es decir se pedirá al usuario que se identifique con su \emph{email y contraseña}. Si no tienen los permisos necesarios se les negará el acceso.

Aquí aparecerán una tabla con la lista de TFGs con estado \emph{Pendiente}, en una tabla con filtros. Ver imagen \ref{fig:tablaPendientes}

\imagenflotante{tablaPendientes}{Tabla con los TFGs con estado Pendiente}{0.9}

En la parte inferior aparecerán una serie de componentes de \emph{Vaadin} para actualizar el estado del TFG que se desee.

\begin{itemize}
	\item \emph{ComboBox} con los títulos del los \emph{TFGs} de la tabla superior para indicar uno. Ver imagen \ref{fig:titulosTFG}
	\item \emph{ComboBox} con las opciones de \emph{Estado} del TFG. Ver imagen \ref{fig:estadosTFG}
	\item Botón para confirmar la modificación del TFG.
\end{itemize}

\imagenflotante{titulosTFG}{Desplegable con los títulos de los TFGs}{0.9}

\imagenflotante{estadosTFG}{Desplegable con los estados del TFG}{0.9}

El \emph{ComboBox} con los estados y el botón para confirmar la modificación no están activados desde el principio, como se puede ver en la figura \ref{fig:desactivados}.

\imagenflotante{desactivados}{Componentes desactivados}{0.9}

Estos componentes se activarán a medida que seleccionemos los datos, es decir, una vez se indique el título se activará la opción de indicar el estado al que se quiere cambiar y una vez indiquemos el estado se activará el botón de actualizar la base de datos.



