\apendice{Especificación de diseño}

\section{Introducción}
En este anexo se detallarán los aspectos referentes al diseño de la aplicación en esta mejora de la aplicación.

\section{Diseño de datos}

\subsection{Ficheros de datos}

En esta nueva versión del proyecto se han añadido varios ficheros de datos nuevos con los que se va a trabajar. Para que se puedan obtener los datos de estos ficheros \textbf{se deben dar unas condiciones}.

\begin{itemize}
	\item El fichero \texttt{src/main/resources/data/BaseDeDatosTFGTFM.xls} se ha modificado, al añadir nuevos datos, con información real sobre proyectos y más peso, surgían \emph{bugs} tratando la información por lo que los campos \emph{fechaAsignacion y fechaPresentacion} en la pestaña \emph{``N3\_Historico''} se han cambiado a tipo \emph{texto} y con el formato estándar europeo DD/MM/AAAA.
	\item El campo \emph{Nota} también se ha modificado, previamente era un número aleatorio entre 5 y 10, se ha sustituido por un valor natural de tipo \emph{general}.
\end{itemize}

En el caso que no se cumpla estas condiciones en las vistas del histórico
y los proyectos activos no se mostraran de manera correcta los datos.

La nueva información obtenida mediante el \emph{webscraping} realizado se ha guardado en dos archivos.
\begin{itemize}
	\item En el archivo csv \emph{``N4\_Profesores''}, contiene en una columna separada por comas los parámetros obtenidos sobre los profesores, nombre y apellidos, área y departamentos.
	\item Esta información se ha añadido en la pestaña \emph{``N4\_Profesores''} de separados por columnas.
\end{itemize}

También se ha creado en una de las nuevas pantallas (generar informe) añadidas un fichero descargable que contiene tres columnas con la información de los profesores de los áreas seleccionadas durante el proceso de creación del proceso, todos los campos que se guardan son tipo \emph{general}.

\subsection{Diagrama de clases}

De forma general la estructura de paquetes y fichero es la misma que en el proyecto anterior de \textbf{\textit{Diana,Gestor-TFG-2021}}~\cite{Gestor-TFG-2021}. Por lo que en este apartado solo se enseñará la estructura de los nuevos ficheros añadidos para las funcionalidades creadas.

\begin{itemize}
	\item En la carpeta \texttt{ubu.digit.ui.view} se han añadido cuatro nuevas clases pertenecientes a las funcionalidades añadidas. Ver imagen \ref{fig:views}
	
	\imagenflotante{views}{Diagrama de clases - Vistas}{0.9}
	
	\item En la carpeta \texttt{ubu.digit.ui.entity} se han otro archivo para representar los proyectos con estado \emph{Pendientes}. Ver imagen \ref{fig:entity}
	
	\imagenflotante{entity}{Diagrama de clases - Entidades}{0.9}
	
\end{itemize}

\section{Diseño procedimental}
En esta versión se han añadido dos nuevas pantallas que requieren de una restricción de usuario para ser utilizadas.
El sistema sigue la siguiente lógica. Ver imagen \ref{fig:diagramaFlujo}

\imagenflotante{diagramaFlujo}{Diagrama de Flujo - Login}{0.9}

\section{Diseño arquitectónico}


