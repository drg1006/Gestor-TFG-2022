\capitulo{3}{Conceptos teóricos}

En este apartado se definirán algunos de los conceptos utilizados a lo largo del proyecto.

\section{\textit{Webscraping}}
El \textbf{\textit{Webscraping}}~\cite{webScrap} es una técnica que se utiliza para la extracción y almacenamiento de información de cualquier página web a través de un programa de software, que suele ser un \textit{crawler}. Aunque, a priori, puede parecer que cualquiera puede scrapear información en cualquier sitio web, lo cierto es que la extracción de datos no siempre es legal. Por ejemplo, los datos que requieren un registro del usuario no pueden ser obtenidos a través del web scraping.

\section{\textit{Desarrollo Backend}}
Dentro del desarrollo web, el \textbf{\textit{backend}}~\cite{backend} se encarga de todos los procesos necesarios para que la web funcione de forma correcta. Estos procesos o funciones no son visibles, pero tienen mucha importancia en el buen funcionamiento de un sitio web. Algunas de estas acciones que controla el backend son la conexión con la base de datos o la comunicación con el servidor de hosting.

\section{\textit{Desarrollo Frontend}}
\textbf{\textit{Frontend}}~\cite{Frontend} es la parte de una aplicación que interactúa con los usuarios, es conocida como el lado del cliente. Básicamente es todo lo que vemos en la pantalla cuando accedemos a un sitio web o aplicación: tipos de letra, colores, adaptación para distintas pantallas (RWD: Responsive Web Design), los efectos del ratón, teclado, movimientos, desplazamientos, efectos visuales… y otros elementos que permiten navegar dentro de una página web. Este conjunto crea la experiencia del usuario.

\section{\textit{Framework}}
Los \textbf{\textit{framework}}~\cite{framework} web son un conjunto de herramientas, estilos y librerías dispuestas a través de una estructura o esqueleto base, para el desarrollo de aplicaciones web más escalables y sencillas de mantener.
Gracias a estos frameworks web, podemos ahorrar grandes cantidades de tiempo y costes, pero vamos a profundizar más en las ventajas que tienen, causantes de su gran éxito y expansión.
\begin{itemize}	
	\item Documentación y comunidad. La cantidad de documentación que podremos encontrar sobre un framework web, suele ser enorme y además con una gran comunidad detrás, respondiendo preguntas y desarrollando nuevas funcionalidades.

	\item Reutilización del código. Uno de los puntos fuertes del los frameworks web es la modularidad de su código y la capacidad para poder hacer múltiples proyectos con el mismo código, cambiando simplemente los textos.

	\item Arquitectura y metodología. La mayoría de frameworks web del mercado usan arquitecturas y metodologías actuales, como el Modelo-Vista-Controlador.

	\item Plantillas web. Las plantillas facilitan mucho el trabajo de los desarrolladores web y los frameworks no se quedan atrás en esto. Algunos frameworks Frontend como \href{https://getbootstrap.com/}{Bootstrap} cuentan con grandes cantidades de plantillas y componentes desarrollados por su extensa comunidad.
	
	\item Seguridad web. Los frameworks web suelen contar con medidas de seguridad para proteger nuestros datos y los de nuestros clientes, ayudando en gran medida en uno de los temas que lleva de cabeza a grandes empresas de servicios web desde el 2017.
	
	\item Posicionamiento en motores de búsqueda. El \href{https://www.illusionstudio.es/servicios/agencia-seo-valencia}{posicionamiento web SEO} on page es muy importante si queremos lograr aparecer en las primeras posiciones de buscadores como Google. Por eso muchos frameworks web ya implementan en su estructura código para poder lograrlo más fácilmente.
	
	\item Ofertas de empleo. Si una persona llega a aprender a usar un framework web de manera solvente, esa persona tendrá muchas posibilidades de encontrar un trabajo más fácilmente y con un sueldo mayor.
\end{itemize}

\section{\textit{ApexCharts}}
\textbf{\textit{Apexcharts}}~\cite{ApexChart}es una biblioteca de gráficos moderna que ayuda a los desarrolladores a crear visualizaciones atractivas e interactivas para páginas web.
Es un proyecto de código abierto con licencia del MIT y es de uso gratuito en aplicaciones comerciales.

\section{\textit{Moodle API}}
\textbf{\textit{Moodlelib API}}~\cite{MoodleAPI} es el archivo de la biblioteca central de diversas funciones de Moodle de uso general. Las funciones se pueden utilizar para el manejo de parámetros de solicitud, configuraciones, preferencias del usuario, tiempo, inicio de sesión, complementos, cadenas y otros. También hay muchas constantes definidas.

Moodle~\cite{Moodle} es una plataforma de aprendizaje diseñada para proporcionarle a educadores, administradores y estudiantes un sistema integrado único, robusto y seguro para crear ambientes de aprendizaje personalizados. 
Moodle es proporcionado gratuitamente como programa de \href{https://opensource.org/docs/osd}{Código Abierto}, bajo la \href{https://docs.moodle.org/dev/License}{Licencia Pública General GNU (GNU General Public License)}. Cualquier persona puede adaptar, extender o Modificar Moodle, tanto para proyectos comerciales como no-comerciales, sin pago de cuotas por licenciamiento, y beneficiarse del costo beneficio, flexibilidad y otras ventajas de usar Moodle. La implementación de Moodle en código abierto significa que Moodle es continuamente revisado y mejorado, para adecuarse a las necesidades actuales y cambiantes de sus usuarios.


