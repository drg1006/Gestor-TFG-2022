\capitulo{3}{Conceptos teóricos}

En este apartado se definirán algunos de los conceptos utilizados a lo largo del proyecto.

\section{\textit{Webscraping}}
El textbf{\textit{WebScraping}}~\cite{webScrap} es una técnica que se utiliza para la extracción y almacenamiento de información de cualquier página web a través de un programa de software, que suele ser un crawler. Aunque, a priori, puede parecer que cualquiera puede scrapear información en cualquier sitio web, lo cierto es que la extracción de datos no siempre es legal. Por ejemplo, los datos que requieren un registro del usuario no pueden ser obtenidos a través del web scraping.

\section{\textit{Desarrollo Backend}}
Dentro del desarrollo web, el \textbf{\textit{backend}}~\cite{backend} se encarga de todos los procesos necesarios para que la web funcione de forma correcta. Estos procesos o funciones no son visibles, pero tienen mucha importancia en el buen funcionamiento de un sitio web. Algunas de estas acciones que controla el backend son la conexión con la base de datos o la comunicación con el servidor de hosting.

\section{\textit{Desarrollo Frontend}}
\textbf{\textit{Frontend}}~\cite{Frontend} es la parte de una aplicación que interactúa con los usuarios, es conocida como el lado del cliente. Básicamente es todo lo que vemos en la pantalla cuando accedemos a un sitio web o aplicación: tipos de letra, colores, adaptación para distintas pantallas(RWD), los efectos del ratón, teclado, movimientos, desplazamientos, efectos visuales… y otros elementos que permiten navegar dentro de una página web. Este conjunto crea la experiencia del usuario.

\section{\textit{Framework}}
Los \textbf{\textit{framework}}~\cite{framework} web son un conjunto de herramientas, estilos y librerías dispuestas a través de una estructura o esqueleto base, para el desarrollo de aplicaciones web más escalables y sencillas de mantener.
Gracias a estos frameworks web, podemos ahorrar grandes cantidades de tiempo y costes, pero vamos a profundizar más en las ventajas que tienen, causantes de su gran éxito y expansión.

\section{\textit{ApexCharts}}
\textbf{\textit{Apexcharts}}~\cite{ApexChart}es una biblioteca de gráficos moderna que ayuda a los desarrolladores a crear visualizaciones atractivas e interactivas para páginas web.
Es un proyecto de código abierto con licencia del MIT y es de uso gratuito en aplicaciones comerciales

\section{\textit{Moodle API}}
\textbf{\textit{Moodlelib API}}~\cite{MoodleAPI} es el archivo de la biblioteca central de diversas funciones de Moodle de uso general. Las funciones pueden superar el manejo de parámetros de solicitud, configuraciones, preferencias del usuario, tiempo, inicio de sesión, mnet, complementos, cadenas y otros. También hay muchas constantes definidas.

