\apendice{Especificación de Requisitos}

\section{Introducción}
Los requisitos\cite{requisitos} del proyecto son los puntos clave que se deben cumplir.

\section{Objetivos generales}
El objetivo del proyecto es la mejora de la aplicación Web centrándose en los siguientes puntos:
\begin{itemize}
	\item Incorporación y validación de un nuevo tipo de datos (XLS)
	\item Realización de la métricas con los resultados del análisis de la calidad del código de los TFG (SonarCloud)
	\item Mejoras en la estética de la aplicación Web.
	\item Modificación del Login para conectarse con la cuenta de la Universidad.
\end{itemize}

\section{Catalogo de requisitos}
Se describirán los requisitos específicos, funcionales y los no funcionales.
\section{Especificación de requisitos}

\subsection{Requisitos funcionales}
\begin{itemize}
	
	\item \textbf{RF-1 Adicción nueva capa de datos}: la aplicación permite emplear dos tipos de archivos, xls y csv, como entrada de datos.
	\begin{itemize}
		\item \textbf{RF-1.1 Subida de datos:} el usuario que tenga permisos podrá subir archivos de datos, tanto en el formato xls como en el csv. 
	\end{itemize}
	
	\item \textbf{RF-2 Login}: el sistema permitirá la subida de datos a ciertos usuarios que se autentificarán a  través del login de la aplicación.
\end{itemize}

\subsection{Requisitos no funcionales}
\begin{itemize}
	\item \textbf{RNF-1 Seguridad}: la aplicación solamente debe permitir la subida de datos a los usuarios con permisos.
	
	\item \textbf{RNF-2 Mantenibilidad}: mejora de la aplicación para permitir la escalabilidad y incorporación de nuevas modificaciones en el futuro de forma sencilla. Un ejemplo de esto sería la actualización de Vaadin a una versión nueva para conservar el soporte en el futuro.
	
	\item \textbf{RNF-2 Mejora diseño}: se realizarán mejoras para que la aplicación sea más atractiva e informativa. Se optará siempre por opciones intuitivas y sencillas de utilizar.

\end{itemize}
