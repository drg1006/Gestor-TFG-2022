\apendice{Especificación de Requisitos}

\section{Introducción}

\textbf{\textit{Los requisitos}}~\cite{Requisito} o requerimientos son las características, las expectativas los aspectos esperados o las capacidades que debe cumplir el producto o servicio que generará el proyecto. Incluye además las cualidades que debe tener el proyecto para cumplir los requisitos del producto. Por esto se distingue entre requisitos funcionales y no funcionales. Los primeros definen qué debe cumplir el producto o servicio y los segundos cómo debe ser el proyecto para que el producto cumpla el propósito.
Solo se incluirán los requisitos realizados en esta mejora.

\section{Objetivos generales}
El principal objetivo de este proyecto es continuar con el desarrollo de la aplicación web respecto a la versión anterior, realizando mejoras gráficas, añadiendo nuevas pantallas y funciones.
Nos centraremos en los siguientes puntos:
\begin{itemize}
	\item Corregir \emph{bugs} o fallos encontrados en la versión previa.
	\item Incorporar la técnica \emph{webscraping} para obtener la información actualizada de los últimos tutores de la Escuela Politécnica Superior.
	\item Implantar una pantalla con un histórico de los tutores, áreas y departamentos de la universidad, atendiendo al número de TFGs asignados por curso académico. 
	\item Generar un informe para el usuario con la información de los tutores, \emph{TFGs} dirigidos, codirigidos y créditos, del área o áreas seleccionados.
	\item Implementar un sistema para que los profesores puedan realizar propuestas de \emph{TFGs} directamente desde la aplicación. Indicando todos los campos que deben tener. 
	\item Incorporar a su vez una entrada para que los administradores puedan aceptar o denegar los \emph{TFGs} que hayan sido propuestos por los tutores, además de implementar una pantalla para modificar los datos de los TFGs activos.
	\item Limitar el acceso a las pantallas dependiendo del rol del usuario que se registra en el sistema.
\end{itemize}

\section{Catálogo de requisitos}
Se describirán los requisitos específicos, funcionales y los no funcionales.
\subsection{Requisitos funcionales}
\begin{itemize}
		\item \texttt{RF-1 Realizar WebScraping}: la aplicación debe obtener los datos de los tutores de la EPS.
		\begin{itemize}
				\item \texttt{RF-1.1 Preguntar al usuario}: preguntar al usuario si desea o no realizar la actualización, indicando la última fecha de modifiación y avisando de que el proceso puede tardar un tiempo.
				\item \texttt{RF-1.2 Búsqueda de datos}: los datos (nombre, apellidos y área) se obtendrán de las \emph{webs}: \href{https://investigacion.ubu.es/unidades/2682/investigadores}{Investigadores} y para obtener el departamento de cada profesor deberemos entrar en \href{https://investigacion.ubu.es/investigadores/34937/detalle}{Detalles}.
				\item \texttt{RF-1.3 Actualizar la información}: guardar los datos en el fichero \emph{BaseDeDatosTFGTFM.xls} y en el fichero \emph{N2 Profesores}, sustituyendo la información previa.
				
		\end{itemize}
		\item \texttt{RF-2  Estadísticas EPS}: mostrar el número de profesores, áreas y departamentos encontrados en la EPS.
		\item \texttt{RF-3 Gráfica histórico profesores}: se creará un gráfico con el histórico de los tutores.
		\begin{itemize}
			\item \texttt{RF-3.1 Parametrizar la búsqueda}: permitir al usuario escoger qué información desea obtener el la gráfica, mostrando todas las opciones de áreas, departamentos y profesores, de las que podrá seleccionar las que desee mostrar.
			\item \texttt{RF-3.2 Obtener el número de TFGs}: mostrar en la gráfica el número de TFGs por curso de los datos seleccionados.
			\item \texttt{RF-3.3 Actualizar gráfica}: darle al usuario la opción de introducir nuevos parámetros y la gráfica se actualizará \emph{clickando} en \emph{Actualizar gráfica}. A su vez, se aleatorizarán los colores que aparecen en la leyenda y las líneas de la gráfica.
		\end{itemize}
		\item \texttt{RF-4 Generar un informe sobre un área}: permitir al usuario seleccionar un área o áreas sobre los que generar un informe con la información  mencionada anteriormente de cada uno de los tutores pertenecientes a ese área. El usuario podrá indicar también el nombre del informe y la hoja de ruta.
		\item \texttt{RF-5 Proponer un TFG}: permitir subir un \emph{TFG} a la base de datos del sistema, introduciendo toda su información mediante \emph{el framework Vaadin}, y además se indica de forma automática que su estado está es \emph{Pendiente de aprobación}.
		\item \texttt{RF-6 Admimistrar los TFGs}: dar la capacidad de visualizar todos los \emph{TFGs} activos para poder modificarlos, y aprobar o denegar los que esten con estado \emph{Pendiente}.
		
\end{itemize}

\subsection{Requisitos no funcionales}
\begin{itemize}
	\item \texttt{RNF-1 Seguridad}: la aplicación deberá verificar que la persona accediendo a las nuevas pantallas, mencionadas en los requisitos RF-4 RF-5 y RF-6, es un usuario con los privilegios requeridos.
	\item \texttt{RNF-2 Mantenimiento y escalabilidad}: permitir que la aplicación siga ordenada, no sea pesada y permita incorporar más funciones a futuro.
	\item \texttt{RNF-3 Documentación}: comentar las modificaciones realizadas para que el usuario que maneje la página sepa cómo se han realizado los procesos. 
	\item\texttt{RNF-3 Mejorar diseño}: se realizarán mejoras gráficas de la aplicación para que resulte más atractiva e informativa. Se optará por opciones intuitivas y sencillas de utilizar, pertenecientes al \emph{framework de Vaadin}.
\end{itemize}

\section{Especificación de requisitos}

\subsection{Diagrama de casos de uso}
En esta sección se mostrarán los diagramas de casos de uso. En la aplicación hay tres actores: alumno, profesor y administrador.
Este rol se le asigna una vez el usuario se registra en la aplicación en la pantalla inicial.


\tablaSmallSinColores{Actores de la aplicación}{p{0.3\textwidth} | p{0.7\textwidth}}{Actores_aplicación}
{\textbf{Usuario} & \textbf{Funcionalidad} \\}{
	Alumno & Es el usuario que menos permisos posee, solo tiene acceso a las pantallas de información general, proyectos activos, histórico de proyectos y métricas. \\\hline
	
	Profesor & Es el usuario con el permiso \emph{reports} para poder generar informes y proponer TFGs, también tendrá acceso a la pantalla de histórico del profesorado, además de tener acceso a todas las mencionadas en el usuario \emph{Alumno}. \\\hline
	
	Administrador & Es el usuario que más permisos tiene, con el permiso \emph{update} que le da el acceso a todas las pantallas/botones restringidos. Tiene la capacidad de administrar los TFGs, es decir, modificar su información, aceptarlos, denegarlos, así como actualizar los datos de las bases de datos utilizadas para guardar la información. \\
}
Se puede ver un resumen de los casos de uso descritos anteriormente en la imagen:---------

\tablaSmallSinColores{Caso de uso 1:  Realizar WebScraping.}{p{3cm} p{.75cm} p{9cm}}{tablaCU1}{
	\multicolumn{3}{p{10.25cm}}{Caso de uso 1:  Realizar WebScraping.} \\
}
{
	Descripción                            & \multicolumn{2}{p{10.25cm}}{Obtención de la información sobre los profesores de la EPS } \\\hline
	Precondiciones                         & \multicolumn{2}{p{10.25cm}}{Las páginas donde se realiza la búsqueda no estén caídas.} \\\hline
	Requisitos                         	   & \multicolumn{2}{p{10.25cm}}{RF-1.1, RF-1.2, RF-1.3} \\\hline
	\multirow{3}{3.5cm}{Secuencia normal}  & Paso & Acción \\\cline{2-3}
	& 1    & Preguntar al usuario si desea actualizar, el usuario pulsará el botón \emph{SI}, si quiere actualizar. \\\cline{2-3}
	Postcondiciones                        & \multicolumn{2}{p{10.25cm}}{Ninguna} \\\hline
	Excepciones                        & \multicolumn{2}{p{10.25cm}}{No se encuentre la ruta del archivo a actualizar o las páginas \emph{web} estén caídas. }\\\hline
	Frecuencia                             & Baja \\\hline
	Importancia                            & Alta \\\hline
	Urgencia                               & Alta \\
}

\tablaSmallSinColores{Caso de uso 2:  Estadísticas EPS.}{p{3cm} p{.75cm} p{9cm}}{tablaCU2}{
	\multicolumn{3}{p{10.25cm}}{Caso de uso 2:   Estadísticas EPS.} \\
}
{
	Descripción                            & \multicolumn{2}{p{10.25cm}}{Mostrar el número de profesores, áreas y departamentos encontrados en la EPS.} \\\hline
Precondiciones                         & \multicolumn{2}{p{10.25cm}}{Ninguna} \\\hline
Requisitos                         	   & \multicolumn{2}{p{10.25cm}}{Ninguna} \\\hline
\multirow{3}{3.5cm}{Secuencia normal}  & Paso & Acción \\\cline{2-3}
& 1    & Acceder a la vista de profesores. \\\cline{2-3}
Postcondiciones                        & \multicolumn{2}{p{10.25cm}}{Ninguna} \\\hline
Excepciones                        & \multicolumn{2}{p{10.25cm}}{Ninguna}\\\hline
Frecuencia                             & Baja \\\hline
Importancia                            & Media \\\hline
Urgencia                               & Baja \\
}

\tablaSmallSinColores{Caso de uso 3:  Gráfica histórico profesores.}{p{3cm} p{.75cm} p{9cm}}{tablaCU3}{
	\multicolumn{3}{p{10.25cm}}{Caso de uso 3: Gráfica histórico profesores.} \\
}
{
	Descripción                            & \multicolumn{2}{p{10.25cm}}{Crear un gráfico con el histórico de los tutores } \\\hline
	Precondiciones                         & \multicolumn{2}{p{10.25cm}}{Ninguna} \\\hline
	Requisitos                         	   & \multicolumn{2}{p{10.25cm}}{RF-3.1, RF-3.2, RF-3.3} \\\hline
	\multirow{3}{3.5cm}{Secuencia normal}  & Paso & Acción \\\cline{2-3}
	& 1    & Indicar los parámetros que el usuario quiere mostrar en la gráfica, profesores, área, departamentos.\\\cline{2-3}
	& 2    & \emph{Clickar} en el boton \emph{Actualizar gráfica}. \\\cline{2-3}
	& 3    & Repetir el proceso para cambiar los parámetros o para añadir otros tutores. \\\cline{2-3}
	Postcondiciones                        & \multicolumn{2}{p{10.25cm}}{Ninguna} \\\hline
	Excepciones                        & \multicolumn{2}{p{10.25cm}}{Ninguna }\\\hline
	Frecuencia                             & Media \\\hline
	Importancia                            & Alta \\\hline
	Urgencia                               & Alta \\
}

\tablaSmallSinColores{Caso de uso 4: Generar un informe.}{p{3cm} p{.75cm} p{9cm}}{tablaCU4}{
	\multicolumn{3}{p{10.25cm}}{Caso de uso 4: Generar un informe.} \\
}
{
	Descripción                            & \multicolumn{2}{p{10.25cm}}{ Crear un informe cuando un usuario seleccione un área o áreas sobre los que obtenerla información mencionada anteriormente de cada uno de los tutores pertenecientes a ese área. } \\\hline
	Precondiciones                         & \multicolumn{2}{p{10.25cm}}{Ninguna} \\\hline
	Requisitos                         	   & \multicolumn{2}{p{10.25cm}}{Ninguno} \\\hline
	\multirow{3}{3.5cm}{Secuencia normal}  & Paso & Acción \\\cline{2-3}
	& 1    & Indicar el área o áreas sobre las que hacer el informe.\\\cline{2-3}
	& 2    & Indicar el número de alumnos matriculados en la asignatura TFG. \\\cline{2-3}
	& 3    & Indicar el nombre que le queremos dar al informe\\\cline{2-3}
	Postcondiciones                        & \multicolumn{2}{p{10.25cm}}{Indicar la ruta en la que descargar el informe generado} \\\hline
	Excepciones                        & \multicolumn{2}{p{10.25cm}}{Ninguna }\\\hline
	Frecuencia                             & Media \\\hline
	Importancia                            & Media \\\hline
	Urgencia                               & Alta \\
}

\tablaSmallSinColores{Caso de uso 5: Proponer un TFG.}{p{3cm} p{.75cm} p{9cm}}{tablaCU5}{
	\multicolumn{3}{p{10.25cm}}{Caso de uso 5: Proponer un TFG.} \\
}
{
	Descripción                            & \multicolumn{2}{p{10.25cm}}{ Permitir a un tutor subir a la base de datos un TFG nuevo. } \\\hline
	Precondiciones                         & \multicolumn{2}{p{10.25cm}}{Ser validado como profesor a través del login} \\\hline
	Requisitos                         	   & \multicolumn{2}{p{10.25cm}}{Ninguno} \\\hline
	\multirow{3}{3.5cm}{Secuencia normal}  & Paso & Acción \\\cline{2-3}
	& 1    & Añadir toda la información en los campos que se indican sobre el \emph{TFG} que se quiera subir titulo, descripción, tutor1,cursoAsignación y tutor2, alumno1, alumno2 si procede.\\\cline{2-3}
	Postcondiciones                        & \multicolumn{2}{p{10.25cm}}{Introducir todos los parámetros que son obligatorios.} \\\hline
	Excepciones                        & \multicolumn{2}{p{10.25cm}}{Ninguna }\\\hline
	Frecuencia                             & Media \\\hline
	Importancia                            & Alta \\\hline
	Urgencia                               & Alta \\
}

\tablaSmallSinColores{Caso de uso 6: Administrar un TFG.}{p{3cm} p{.75cm} p{9cm}}{tablaCU6}{
	\multicolumn{3}{p{10.25cm}}{Caso de uso 6: Administrar un TFG.} \\
}
{
	Descripción                            & \multicolumn{2}{p{10.25cm}}{Permitir a un administrador modificar los datos de los TFGs activos además de cambiar el estado de los que están en \emph{Pendiente} } \\\hline
	Precondiciones                         & \multicolumn{2}{p{10.25cm}}{Ser validado como administrador a través del login} \\\hline
	Requisitos                         	   & \multicolumn{2}{p{10.25cm}}{Ninguno} \\\hline
	\multirow{3}{3.5cm}{Secuencia normal}  & Paso & Acción \\\cline{2-3}
	& 1    & Sustituir el campo estado del \emph{TFG} que se quiera modificar por \emph{Aceptado o Denegado}.\\\cline{2-3}

	Postcondiciones                        & \multicolumn{2}{p{10.25cm}}{Comprobar que los datos se han introducido de manera correcta y que no falte ninguno de los parámetros obligatorios.} \\\hline
	Excepciones                        & \multicolumn{2}{p{10.25cm}}{Ninguna }\\\hline
	Frecuencia                             & Media \\\hline
	Importancia                            & Alta \\\hline
	Urgencia                               & Alta \\
}
