\apendice{Especificación de Requisitos}

\section{Introducción}
La especificación de requisitos hace referencia a los requerimientos que debe cumplir el software para satisfacer las necesidades del cliente. Debe incluir la suficiente cantidad de detalles para permitir a los desarrolladores software diseñar el sistema.

\section{Objetivos generales}
El objetivo general del proyecto es continuar con el desarrollo y la mejora de la aplicación web, Gestor de Trabajos de Fin de Grado, centrándose en los siguientes puntos:
\begin{itemize}
	\item Agregar y validar un nuevo tipo de datos, Microsoft Excel(xls).	
	\item Analizar la calidad del código de los Trabajos de Fin de Grado de años anteriores empleando la plataforma SonarCloud.
	\item Mejorar la estética de la aplicación Web.
	\item Incorporar un sistema de autenticación de los usuarios al loguearse.
	\item Añadir nuevos valores estadísticos sobre los Trabajos de Fin de Grado. 
\end{itemize}

\section{Catalogo de requisitos}
Se describirán los requisitos específicos, funcionales y los no funcionales.

\subsection{Requisitos funcionales}
\begin{itemize}
	
	\item \textbf{RF-1 Autenticación de usuarios}: la aplicación debe permitir comprobar la identidad del usuario.
	\begin{itemize}
		\item \textbf{RF-1.1 Verificación de la identidad del usuario:} se comprobará la identidad del usuario al intentar acceder al login.
		\item \textbf{RF-1.2 Chequeo de permisos del usuario:} la aplicación revisará si el usuario posee permisos de actualización en la aplicación.
		\item \textbf{RF-1.2 Conceder el acceso al usuario:} se deberá aprobar la entrada del usuario autentificado a las páginas restringidas.
	\end{itemize}

	\item \textbf{RF-2 Actualización de ficheros xls}: la aplicación permite emplear dos tipos de archivos, xls y csv, como entrada de datos.
	\begin{itemize}
		\item \textbf{RF-2.1 Subida de datos:} el usuario autentificado podrá subir archivos de datos, tanto en el formato xls como en el csv. 
		\item \textbf{RF-2.2 Validación de los datos:} se permitirá subir unicamente los ficheros en el formato establecido.
		\item \textbf{RF-2.3 Actualización de la información:} la aplicación deberá actualizar los datos con los existentes en el fichero subido.
	\end{itemize}

	\item \textbf{RF-3 Introducción de nuevas estadísticas}: se incluirá nueva información acerca de los TFGs realizados anteriomente.
	\begin{itemize}
		\item \textbf{RF-2.1 Mostrar rankings de notas:} se mostrarán las clasificaciones de los proyectos realizados a modo de rankings.
	\end{itemize}
		
\end{itemize}

\subsection{Requisitos no funcionales}
\begin{itemize}
	\item \textbf{RNF-1 Seguridad}: la aplicación solamente debe permitir la subida de datos a los usuarios con permisos.
	
	\item \textbf{RNF-2 Mantenibilidad}: mejora de la aplicación para permitir la escalabilidad y incorporación de nuevas modificaciones en el futuro de forma sencilla. 
	
	\item \textbf{RNF-2 Mejora diseño}: se realizarán mejoras gráficas de la aplicación para que resulte más atractiva e informativa. Se optará siempre por opciones intuitivas y sencillas de utilizar.
	
	\item \textbf{RNF-3 Analizar la calidad de código}: cómo parte del apartado de las Métricas se examinaran Trabajos de Fin de Grado de años anteriores, con el fin de añadir más información acerca de ellos.

\end{itemize}

\section{Especificación de casos de uso}