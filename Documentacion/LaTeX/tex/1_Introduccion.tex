\capitulo{1}{Introducción}

El proyecto se centrará en la mejora de la actual aplicación del Gestor de asignaciones de TFG, utilizada en el grado de Ingeniería Informática. La aplicación actual se trata de una actualización previa del TFG denominado \href{https://github.com/dbo1001/Gestor-TFG-2021}{GII 20.09 Herramienta web
repositorios de TFGII}. 



\section{Estructura de la memoria}
La memoria consta de los siguientes apartados:

\begin{itemize}
	\item \textbf{Introducción:} presentación del proyecto y estructuración de este.
	\item \textbf{Objetivos del proyecto:}  exposición de los  objetivos generales, técnicos y personales del proyecto.
	\item \textbf{Conceptos teóricos:} explicación de los términos teóricos necesarios para la comprensión y el desarrollo del proyecto.
	\item \textbf{Técnicas y herramientas:} definición de las técnicas utilizadas para el desarrollo del proyecto y de las herramientas empleadas para su funcionamiento.
	\item \textbf{Aspectos relevantes del desarrollo:} breve explicación de los términos más importantes durante el desarrollo del proyecto.
	\item \textbf{Trabajos relacionados:} descripción de los trabajos y proyectos asociados con la gestión de trabajos de fin de grado (TFG).
	\item \textbf{Conclusiones y líneas de trabajo futuras:} conclusiones obtenidas al finalizar el proyecto y descripción de posibles futuras líneas de trabajo o mejoras.
	\item \textbf{Anexo}: se realiza de forma complementaria un documento de Anexo con explicación más detallada de todo el desarrollo del proyecto.
\end{itemize}