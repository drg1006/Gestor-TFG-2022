\capitulo{5}{Aspectos relevantes del desarrollo del proyecto}

Este apartado recoge los aspectos más interesantes del desarrollo del proyecto.

\section{Problemas a la hora de ejecutar el proyecto}

A la hora de ejecutar el proyecto anterior surgieron una serie de problemas tanto para la ejecución por terminal como desde Eclipse.

Cuando quise ejecutarlo mediante la terminal desplegando el archivo <<.war>> generado tras compilar me surgía el siguiente error un error de conexión que solo indicaba que no se podía acceder al sitio \emph{web}.

Tras buscar información sobre el posible error, se descubre en los logs que proporciona tomcat lo siguiente, ver imagen \ref{fig:error_size}. En el que se informa que se intenta ejecutar un proyecto con un tamaño mayor al que tenemos configurado en tomcat.

\imagenflotante{error_size}{Logs proporcionados por Tomcat}{1.8}

Para solucionar este problema se accede al archivo \emph{apache-tomcat-9.0.68-webapps-manager-WEB-INF} y se modifican las líneas que se ven en la imagen \ref{fig:size_tomcat} aumentando el número que se indica.

\imagenflotante{size_tomcat}{Logs proporcionados por Tomcat}{0.9}

Cuando quise ejecutarlo mediante Eclipse no era posible añadir el proyecto al servidor de \emph{tomcat}, indicando que las versiones no eran compatibles. Por ello se ha entrado en las propiedades del proyecto y se ha cambiado la versión del parametro \emph{Dynamic Web Module} a la 3.1 en el apartado \emph{Project Facets} como se aprecia en la imagen \ref{fig:Dynamic}.

\imagenflotante{Dynamic}{Cambio de versión Dynamic Web Module}{0.9}

\section{WebScraping}

Uno de los objetivos del proyecto que teníamos era la obtención de datos sobre los profesores que se encuentran en la EPS mediante \emph{webscraping}, realizando finalmente con la librería \href{https://jsoup.org/}{JSoup}, tras debatir su uso con otras librerías planteadas. 
Con ella hemos sacado los datos: nombre, apellidos, área y departamento de cada uno de los profesores. Una vez obtenemos esta información la utilizamos para actualizar a la última versión las bases de datos que utilizamos durante todo el proyecto.
También se ha utilizado la librería \href{https://opencsv.sourceforge.net/apidocs/com/opencsv/CSVWriter.html}{CSVWriter} y la \emph{API}: \href{https://poi.apache.org/apidocs/dev/org/apache/poi/ss/usermodel/WorkbookFactory.html}{WorkbookFactory} para escribir los datos en los ficheros correspondientes.

\section{Trabajo con Vaadin}

Este proyecto se ha tenido que llevar a cabo con \emph{Vaadin}, es la primera vez que se utiliza este \emph{framework}, por lo que se ha tenido que estudiar todo su funcionamiento, y cómo implantarlo durante el desarrollo.

Se han tenido que utilizar muchos componentes nuevos que no estaban en la versión anterior como son:

\begin{itemize}
	\item \emph{TextArea}: componente para guardar texto introducido por teclado por parte del usuario.
	\item \emph{Checkbox}: componente que permite realizar una selección múltiple de elementos.
	\item \emph{Binder}: componente para conectar la clase del formulario con la información requerida, y poder asignar parámetros obligatorios para rellenar.
	\item \emph{ComboBox}: componente desplegable con diferentes datos.
	\item \emph{Anchor}: componente utilizado para realizar el documento descargable.
	\item \emph{Dialog}: componente estilo \emph{pop-up} para realizar confirmaciones.
	\item \emph{DatePicker}: componente que permite seleccionar una fecha a través de un calendario.

\end{itemize}

Toda la información sobre el uso de \emph{Vaadin} \url{https://vaadin.com/} se ha obtenido de su página atendiendo a la versión que se utiliza.

\section{Actualización de ficheros}

Se ha tenido que incorporar en varias ocasiones la creación de archivos \emph{.xls o .csv}, así como la actualización en tiempo real de la información de la base de datos, ya sea al introducir un TFG, al modificar su estado, o cuando queremos actualizar la información del \emph{webscraping} sobre los tutores.

Para ello se han empleado las librerías \emph{CSVWriter o WorkBook}, estudiando su funcionamiento.

\section{Validación de los usuarios de la aplicación}

En esta versión se ha añadido la posibilidad de iniciar sesión en varios dispositivos o navegadores al mismo tiempo, de esta forma también se gestinan los roles en la aplicación según el tipo de usuario que haya iniciado sesión. Existen tres roles dependiendo de los permisos que tengan en la asignatura \emph{Trabajos de Fin de Grado} en \emph{Moodle}, es decir, que para saber el rol de un usuario es necesario conectarnos al \emph{Moodle} de la \emph{UBU} para dicha asignatura.

Cuando se inicie sesión, de manera opcional, se mostrarán diferentes opciones en la barra de navegación para los roles que están especificados: alumno, profesor no editor y administrador.