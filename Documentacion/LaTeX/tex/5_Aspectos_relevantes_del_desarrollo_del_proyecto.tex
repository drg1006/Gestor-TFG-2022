\capitulo{5}{Aspectos relevantes del desarrollo del proyecto}

Este apartado recoge los aspectos más interesantes del desarrollo del proyecto.

\section{WebScrap}

Uno de los objetivos que teníamos era la obtención de datos sobre los profesores que se encuentran en la EPS mediante \emph{webscraping}, realizando finalmente con la librería \href{https://jsoup.org/}{JSoup}, tras debatir su uso con otras librerías planteadas. 
Con ella hemos sacado los datos: nombre, apellidos, área y departamento de cada uno de los profesores. Una vez obtenemos esta información la utilizamos para actualizar a la última versión las bases de datos que utilizamos durante todo el proyecto.
También se ha utilizado la librería \href{https://opencsv.sourceforge.net/apidocs/com/opencsv/CSVWriter.html}{CSVWriter} y la \emph{API}: \href{https://poi.apache.org/apidocs/dev/org/apache/poi/ss/usermodel/WorkbookFactory.html}{WorkbookFactory} para escribir los datos en los ficheros correspondientes, ya definidas en el Anexo.

\section{Implementación de la pantalla de histórico del profesorado}
Otro objetivo que teníamos era presentar una pantalla con los datos del profesorado durante los últimos cursos. 
Para ello primero hemos recolectado toda la información sobre los tutores y le hemos dado al usuario la opción de seleccionar los parámetros (areas, departamentos y profesores) que quiera añadir en la gráfica en la que se muestra el número de TFGs realizados durante los últimos cursos.
Para ello hemos utilizado el \emph{framework} de \href{https://vaadin.com/}{Vaadin} que nos permite mostrar de una forma más interactiva los datos que hemos obtenido al realizar el \emph{webscraping}.

Primero le damos la opción de actualizar a la última versión los datos que vamos a utilizar para representar la gráfica, informándole de la última versión de los datos y advirtiendo de que el proceso puede llevar un tiempo (aproximadamente un minuto). Ver imagen \ref{fig:ActualizarDatos}

\imagenflotante{ActualizarDatos}{Preguntar al usuario si quiere actualizar la base de datos}{0.9}
Mostramos también información sobre el número total de profesores, áreas y departamentos encontrados realizados mediante consultas \emph{SQL}.

Posteriormente tenemos dos \emph{CheckBoxes} donde le hemos dado al usuario la capacidad de seleccionar tantas áreas como departamentos desee mostrar en el gráfico y un \emph{ComboBox} con los nombres de los profesores, donde podremos introducir uno a uno los tutores que queremos que se muestren. Finalmente tras pulsar en el boton \emph{Actualizar gráfica} se nos mostrarán los TFGs dirigidos por area, departamento o profesor seleccionados y con unos colores aleatorios también implantados.Ver imagen \ref{fig:Grafica}

\imagenflotante{Grafica}{Gráfica con el número de TFGs por curso seleccionados}{0.9}

\section{Implementación de la pantalla de generación de informes}

En esta pantalla lo que se trata de crear es un informe en el que se muestren el número de TFGs dirigidos, codirigidos y el número de créditos asignados a los profesores durante el último curso académico de un área determinado por el usuario.
Para ello le damos al usuario la opción de indicar las áreas que desea analizar y le pedimos que indique un nombre para el mismo. Ver imagen \ref{fig:informe}.

\imagenflotante{informe}{Opciones de los informes}{0.9}

\section{Asignación de TFGs}

Se añadirá a medida que se vaya implementando.