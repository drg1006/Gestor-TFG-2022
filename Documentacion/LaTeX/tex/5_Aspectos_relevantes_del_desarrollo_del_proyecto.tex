\capitulo{5}{Aspectos relevantes del desarrollo del proyecto}

Este apartado recoge los aspectos más interesantes del desarrollo del proyecto.

\section{WebScrap}

Uno de los objetivos que teníamos era la obtención de datos sobre los profesores que se encuentran en la EPS mediante \emph{webscraping}, realizando finalmente con la librería \href{https://jsoup.org/}{JSoup}, tras debatir su uso con otras librerías planteadas. 
Con ella hemos sacado los datos: nombre, apellidos, área y departamento de cada uno de los profesores. Una vez obtenemos esta información la utilizamos para actualizar a la última versión las bases de datos que utilizamos durante todo el proyecto.
También se ha utilizado la librería \href{https://opencsv.sourceforge.net/apidocs/com/opencsv/CSVWriter.html}{CSVWriter} y la \emph{API}: \href{https://poi.apache.org/apidocs/dev/org/apache/poi/ss/usermodel/WorkbookFactory.html}{WorkbookFactory} para escribir los datos en los ficheros correspondientes, ya definidas en el Anexo.

Este recurso es muy utilizado para obtener información de \emph{páginas webs} externas, sin embargo, hay que tener cuidado con las páginas de las que se obtienen los datos. Por ejemplo, los datos que requieren un registro del usuario no pueden ser obtenidos a través del \emph{web scraping}.

Tampoco es legal el uso de esta técnica de rastreo para ocultar publicidad, descargas de responsabilidad o términos y condiciones.~\cite{webScrap}

\section{Trabajo con vaadin}

Este proyecto se ha tenido que llevar a cabo con \emph{Vaadin}, es la primera vez que se utiliza este \emph{framework}, por lo que se ha tenido que estudiar todo su funcionamiento, y como implantarlo durante el desarrollo.

Se han tenido que utilizar muchos componentes nuevos que no estaban en la versión anterior como pueden ser:

\begin{itemize}
	\item \emph{TextArea}: componente para guardar texto introducido por teclado por parte del usuario.
	\item \emph{Checkbox}: componente que permite realizar una selección múltiple de elementos.
	\item \emph{Binder}: componente para conectar la clase del formulario con la información requerida.
	\item \emph{ComboBox}: componente desplegable con diferentes datos.
	\item \emph{Anchor}: componente utilizado para realizar el documento descargable.
	\item \emph{Dialog}: componente estilo \emph{pop-up} para realizar confirmaciones.

\end{itemize}

Toda la información sobre el uso de \emph{Vaadin} \url{https://vaadin.com/} se ha obtenido de su página atendiendo ala versión que se utiliza.

\section{Actualización de ficheros}

Se ha tenido que incorporar en varias ocasiones la creación de archivos \emph{.xls o .csv}, así como la actualización en tiempo real de la información de la base de datos, ya sea al introducir un TFG, al modificar su estado, o cuando queremos actualizar la información del \emph{web scraping} sobre los tutores.

Para ello se ha empleado las librearías \emph{CSVWriter o WorkBook}, estudiando su funcionamiento.
