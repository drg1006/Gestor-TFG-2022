\capitulo{2}{Objetivos del proyecto}

Este apartado explica de forma precisa y concisa cuales son los objetivos que se persiguen con la realización del proyecto.

\section{Objetivos generales}
\begin{itemize}
	\item Corregir \emph{bugs} de la versión anterior.
	\item Crear nuevas pantallas para completar la aplicación.
		\begin{itemize}
			\item Pantalla con la visualización del histórico de los profesores/áreas/departamentos.
			\item Permitir crear un informe de representación de datos de profesores y áreas.
			\item Implementar la opción de oferta/asignación de TFGs.
			\item Implementar la funcionalidad de aceptar/denegar TFGs propuestos, así como de modificar los proyectos de la pestaña de activos.
		\end{itemize}

	
\end{itemize}
\section{Objetivos técnicos}
\begin{itemize}
	\item Implementar \emph{webscraping} con la página \href{https://investigacion.ubu.es/unidades/2682/investigadores}{de investigadores de la UBU}.
	\item Permitir la oferta de proyectos desde la propia aplicación siendo profesor y la aceptación de los mismos por parte de un administrador.
	\item Crear una nueva gráfica para las estadísticas de los profesores con las opciones marcadas por el usuario.
	\item Guardar datos generados en archivos \emph{csv} y \emph{xls}.
	\item Actualizar los ficheros de datos del profesorado.
	\item Utilizar GitHub para llevar a cabo el seguimiento del proyecto y control de versiones.
\end{itemize}