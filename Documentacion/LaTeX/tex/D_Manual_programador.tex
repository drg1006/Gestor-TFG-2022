\apendice{Documentación técnica de programación}

\section{Introducción}
En esta sección se van a detallarlos diferentes procesos de instalación de las herramientas que se han utilizado durante el proyecto. También se especificará la estructura del proyecto, instalación de dependencias, la compilación, la ejecución del proyecto y el despliegue en Heroku. 
Algunos detalles son similares a los utilizados en el anterior proyecto de Diana, \textbf{\textit{Gestor-TFG-2021}}~\cite{Gestor-TFG-2021}.

\section{Estructura de directorios}
Se enumerarán y describirán brevemente los directorios del proyecto. Se puede encontrar el código fuente en el repositorio de Github denominado \href{https://github.com/drg1006/gestor-tfg-2022}{``Gestor-TFG-2022''}. También se indicarán los nuevos archivos añadidos en esta versión.

\begin{itemize}
	\item \texttt{/:} directorio raíz donde se ubican el README, Maven. 
	\item \texttt{/.github/workflows} los archivos de \textit{workflow} o flujo de trabajo, para la Integración continua del proyecto en GitHub.
	\item \texttt{/Documentacion} material de documentación del proyecto y prueba empleadas.
	\begin{itemize}
		\tightlist
		\item \texttt{/Documentacion/LaTeX} ficheros para generar la memoria y los anexos realizados en \emph{TexStudio}.
		\item \texttt{/Documentacion/Pruebas} aplicaciones prototipo para comenzar el aprendizaje con \href{https://vaadin.com/}{Vaadin} y pruebas realizadas con diferentes librerías durante el Web Scraping.
	\end{itemize}
	\item \texttt{/frontend} código encargado del diseño gráfico de la aplicación por el lado del cliente.
	\item \texttt{/src}, estructura de directorios \emph{backend} de la aplicación. Ya explicada en la versión anterior del proyecto:
		\begin{itemize}
			\item \texttt{/src/main/java/ubu/digit/persistence} código fuente encargado de la conexión y lectura de los ficheros de datos (fachada de datos).		
			\item \texttt{/src/main/java/ubu/digit/security} código fuente de conexión y consulta con el moodle de UbuVirtual.
			\item \texttt{/src/main/java/ubu/digit/ui} código en relación a las ventanas y vistas de la aplicación.
			\begin{itemize}
				\item \texttt{/src/main/java/ubu/digit/ui/entity} código con las entidades de los proyectos, cursos y usuarios de la aplicación.
					\begin{itemize}		
						\item \texttt{FormularioTFG.java}, nueva entidad con los campos del formulario del TFG.
					\end{itemize}
				\item \texttt{/src/main/java/ubu/digit/ui/components} código con la interfaz gráfica de la barra de navegación y la de pie de página.
				\item \texttt{/src/main/java/ubu/digit/ui/views} código con las vistas de la aplicación.
					\begin{itemize}
						\item \texttt{ProfesoresView.java}, nueva vista con la información histórica de los profesores.
						\item \texttt{ReportView.java}, nueva vista con el código para la realización de reportes.
						\item \texttt{newProjectView.java}, nueva vista con el código que permite subir propuestas de TFGs al servidor.
						\item \texttt{ManageView.java}, nueva vista con el código que permite aceptar o denegar las propuesta de TFGs con estado \emph{Pendiente}, además de permitir modificar los que están activos.
						\item \texttt{ModifyView.java}, nueva vista con el código que permite modificar los datos de un TFG seleccionado previamente en la vista de ManageView.
					\end{itemize}
				\item \texttt{/src/main/java/ubu/digit/util} incluye los métodos empleados de utilidad empleados en toda la app. 
				\item \texttt{/src/main/java/ubu/digit/webService} servicios web empleados para la consulta en moodle.
			\end{itemize}
			
			\item \texttt{/src/test} tests unitarios sobre las clases fachada “SistInfDataCsv” y “SistInfDataXls”.
			
			\item \texttt{/src/main/resources} carpeta con los datos que se van a cargar en el servidor para obtener toda la información necesaria.
		\end{itemize}
	
\end{itemize}
			
\section{Manual del programador}

A continuación se detallará el proceso de instalación de los programas
necesarios para el desarrollo de la aplicación.

\subsection{Instalación de Java}

Actualmente se sigue ejecutando con la versión de Java 11.

Para ello se debe descargar la \href{https://www.oracle.com/es/java/technologies/javase/jdk11-archive-downloads.html}{página de descargas de Oracle Java SE 11.0} y descargar la versión de JDK 11, correspondiente con el sistema operativo que se posea y su arquitectura, ya sea de 64 o 32 bits. Ver imagen \ref{fig:Descarga_JDK_11}.

Tras escoger la versión según el SO, se leerán y aceptarán las licencias de uso de Oracle \ref{fig:Descarga_JDK11_Licencia}, y se hará \emph{click} en descargar.

\imagenflotante{Descarga_JDK_11}{Descarga de JDK 11}{0.9}

También se deberá cambiar la variable de entorno de Java del sistema.

\imagenflotante{Descarga_JDK11_Licencia}{Descarga JDK 11 Licencia}{0.9}


\subsection{Instalación de Eclipse}
A continuación se instalará un entorno de desarrollo integrado(IDE) para Java, en este caso se ha utilizado \textbf{Eclipse IDE for Enterprise Java Developers} en la versión 2021-12. 

Para descargar el IDE se accederá a la \href{https://www.eclipse.org/downloads/packages/release/2021-12/r}{página de descargas de Eclipse} y descargar la opción correspondiente a nuestro sistema operativo del \textbf{Eclipse Installer 2021-12 R}. 

En el caso de los sistemas operativos Windows se descargará un archivo ejecutable que se deberá ejecutar como administrador. Una vez ejecutado se deberá seleccionar la opción ``\textbf{\textit{Eclipse IDE for Enterprise Java Developers}}''. Ver imagen \ref{fig:Eclipse_Installer}. 

\imagenflotante{Eclipse_Installer}{Seleccionar Eclipse}{0.9}

Por último seleccionaremos el JDK (11) que vayamos a utilizar y la carpeta donde queremos instalar nuestro IDE.

\subsection{Instalación del \textit{plugin de Vaadin} para Eclipse}
Una vez se haya instalado Eclipse, se procederá a añadir el plugin de Vaadin para Eclipse. Esto se realizará mediante el \textbf{Eclipse Marketplace de Eclipse} \ref{fig:Eclipse_marketplace}, el cual se encuentra en la opción de ``\textbf{\textit{Help/Eclipse Marketplace...}}'' de la barra de herramientas.

\imagenflotante{Eclipse_marketplace}{Eclipse marketplace}{0.7}

Una vez en el Eclipse Marketplace, se buscará ``\textbf{Vaadin}'' y se pulsará ``\textbf{Go}''. Tras salir el plugin ``\textbf{\textit{Vaadin Plugin for Eclipse}}'', se pulsará a ``\textbf{Install}'' y comenzará la instalación del plugin \ref{fig:Plugin_Vaadin}. En la imagen ya se muestra una vez instalado.

\imagenflotante{Plugin_Vaadin}{Plugin Vaadin}{0.9}

\section{Compilación, instalación y ejecución del proyecto}

Se explicará como compilar, instalar y ejecutar el proyecto. En el caso de
la ejecución, se detallará como hacerlo desde un terminal y mediante Eclipse (IDE).

\subsection{Descarga del repositorio}
El código fuente se encuentra en el \href{https://github.com/drg1006/Gestor-TFG-2022.git}{repositorio del proyecto} en GitHub. Para descargarlo se deberá hacer click en ``\textbf{\textit{Code}}'' y copiar la URL que aparece en el apartado de ``\textbf{HTTP}''. Con esta URL deberemos ir al ``\textbf{GitHub Desktop}'' y clonar el repositorio \ref{fig:GitHub_Code}.

\imagenflotante{GitHub_Code}{Copiar URL repositorio}{0.9}

Si se desea tener código en local se deberá descargar el zip ``\textbf{\textit{Download ZIP}}'' en la opción ``\textbf{\textit{Code}}'' anteriormente mencionada. Una vez descargado el fichero se descomprimirá y abrirá con Eclipse. 

\subsection{Compilación del proyecto}

Para compilar el proyecto en local desde terminal se usará:
\begin{itemize}
	\item Limpiar las dependencias: \texttt{mvn clean}.
	\item Instalar dependencias y compilar: \texttt{mvn install}.
	\item Instalar en modo producción (para desplegar): \texttt{mvn package -Pproduction}.  
	\item Ejecutar test: \texttt{mvn test}.	
\end{itemize}

\subsection{Ejecución del proyecto desde local}

Para la ejecución del proyecto en local desde terminal se usará:
\begin{itemize}
	\item Entrar en la terminal que utilicemos.
	\item Acceder a la carpeta donde tenemos nuestro servidor tomcat instalado y entrar en la carpeta \textbf{/bin}.
	\item Ejecutar nuestro servidor local mediante \textbf{startup} \ref{fig:tomcat_cmd}.  
	\item Entrar en el nuestro navegador en la direccion \textbf{localhost:8080}.
	\item Pulsar en la opción Manage App \ref{fig:Manage_App}.
	\item Iniciamos sesión como manager-gui.(Indicado en el archivo /conf/tomcat-users.xml).
	\item Llegaremos a la pantalla \ref{fig:Desplegar_war} y seleccionaremos el archivo .war que hemos creado al compilar nuestro proyecto con \textbf{``mvn package -Pproduction''}.
	
	\imagenflotante{tomcat_cmd}{Consola con Tomcat ejecutado}{0.9}
	
	\imagenflotante{Manage_App}{Gestor de Aplicaciones de Tomcat}{0.9}
	
	\imagenflotante{Desplegar_war}{Desplegar el archivo .war}{0.9}
	
\end{itemize}

\subsection{Ejecución del proyecto desde Eclipse IDE}

Para la ejecución del proyecto en local desde Eclipse primero debemos importar como proyecto Maven, con el \textbf{pom.xml}. Utilizaremos también un servidor local de de \href{https://tomcat.apache.org/i}{Apache Tomcat}, en concreto, la versión 9. Se puede descargar en \href{https://tomcat.apache.org/download-90.cgi}{la página oficial de Apache Tomcat}.


Una vez descargado y descomprimido, se creará un servicio de Tomcat, ver imagen \ref{fig:añadirTomcat}, con la ruta donde se tiene descargado Tomcat y se le dará un nombre, ver imagen \ref{fig:nombreServerTomcat}. Por último, se añadirá el proyecto principal ``sistinf'', ver imagen \ref{fig:tomcatAñadirProyecto}.

\imagenflotante{añadirTomcat}{Añadir servidor de Tomcat a Eclipse}{0.9}
\imagenflotante{nombreServerTomcat}{Seleccionar carpeta contenedora de Tomcat}{0.9}
\imagenflotante{tomcatAñadirProyecto}{Añadir proyectos a servidor}{0.9}

Para ejecutarlo desde Eclipse debemos también seguir todos los pasos de compilación anteriormente mencionados.

Una vez tengamos compilado nuestro código debemos ejecutarlo (click derecho en el proyecto $\rightarrow$ \texttt{Run As} $\rightarrow$ \texttt{Run on Server}).

Si no aparece la vista de los servicios se puede añadir desde la barra de herramientas$\rightarrow$ \texttt{Window} $\rightarrow$ \texttt{Show View}. Para configurar la ruta donde se ejecuta la aplicación, por defecto en \href{http://localhost:8080/}{localhost:8080/} o en ciertos casos \href{http://localhost:8080/sistinf}{localhost:8080/sistinf}.

\subsection{Alternativa a Heroku}

A partir del día 28 de noviembre Heroku dejará de ser gratuito, pero ofrecen una alternativa para estudiantes. Esta opción es un acuerdo entre el \href{https://education.github.com}{programa de estudiantes de GitHub} y Heroku. 
Esta colaboración viene explicada en la plataforma de \href{https://blog.heroku.com/github-student-developer-program}{Heroku}.

Por ello mismo migraremos nuestro proyecto a esa versión de Heroku ya que GitHub estudiantes nos proporciona una serie de créditos mensuales durante un año para desplegar nuestra aplicación.

Los pasos a seguir son los siguientes:
\begin{itemize}
	\item Obtener la verificación GitHub para estudiantes en
	 \href{https://education.github.com/students}{https://education.github.com/students}, ver imagen\ref{fig:GitHub_stu}.
	\item Seguir las instrucciones tras entrar en \emph{Sign up for Global Campus} como se indica en \ref{fig:verify_GH}.
	\item Indicar la escuela/universidad a la que pertenecemos y el uso que le vamos a dar a la cuenta, ver imagen \ref{fig:parametros_GH}.
\end{itemize}

\imagenflotante{GitHub_stu}{Página de acceso a GitHub estudiantes}{0.9}

\imagenflotante{verify_GH}{Proceso de petición de GitHub for students}{0.9}

\imagenflotante{parametros_GH}{Aportar la información necesaria para la verificación}{0.9}

Tras ser verificados por GitHub debemos conectar nuestra cuenta con Heroku desde \href{https://www.heroku.com/github-students}{https://www.heroku.com/github-students}.

Los pasos a seguir son los siguientes:
\begin{itemize}
	\item Conectar nuestra cuenta de Heroku con la cuenta de GitHub para estudiantes.
	\item Debemos añadir una tarjeta bancaria, ya que vamos a utilizar un servicio de pago pero de forma gratuita, por lo que se solicitan esos datos (que se podrán retirar en un futuro sin ningún tipo de pago realizado).
	\item Esperar a que se confirme la solicitud realizada.
	\item Podemos comprobar si hemos sido verificados si tenemos los créditos añadidos en nuestra pestaña de \emph{Account Settings} en \emph{Billing} \ref{fig:Billing}.
\end{itemize}
El resultado final debería ser el siguiente \ref{fig:final_HGH}
\imagenflotante{final_HGH}{Pestaña final tras aplicar a la oferta}{0.9}

\imagenflotante{Billing}{Créditos de nuestra cuenta}{0.9}

Una vez hemos obtenido los créditos modificamos el plan de nuestra aplicación y cambiamos al plan \emph{Basic} con un precio de 7 dólares al mes. Ver imagen \ref{fig:Basic}

\imagenflotante{Basic}{Modificar el plan de despliegue.}{0.9}

\section{Pruebas del sistema}

En esta sección se detallarán todas las pruebas realizadas en la aplicación para comprobar su correcto funcionamiento.

Se realizarán pruebas manuales para todas las nuevas funcionalidades añadidas en esta versión.

\subsection{Archivo de prueba: Prueba01}
Este archivo situado en la carpeta \texttt{Documentacion/Pruebas} denominado \texttt{Prueba01.xls}, es un archivo estructurado de igual forma que la base de datos que utilizamos de forma habitual pero con menos carga de datos.
Se cargará con la opción \emph{Actualizar base de datos} de la parte inferior de la pantalla, de acceso exclusivo para los administradores, y se probará su correcto tratamiento durante todas las pantallas de la aplicación.

Tiene un total de nueve proyectos en la pestaña de activos, tres de ellos con un estado \emph{Pendiente} y un total de 10 en la de históricos.

\begin{enumerate}
	\item \textbf{Proyectos activos}: en esta pantalla se muestran las estadísticas sobre los proyectos de la pestaña \texttt{N2 Proyecto}. Los resultados que debemos esperar son los siguientes:
		\begin{enumerate}
			\item 6 proyectos activos, ya que habiendo un total de 9, 3 de ellos son con estado \emph{Pendiente}, por lo que no se contarán.
			\item 4 proyectos sin asignar, ya que 2 de los 6 tienen alumno asignado.
			\item 17 alumnos sin proyecto asignado, obtenidos en la pestaña \texttt{N2 Alumno}
			\item 8 tutores asignados, ya que el tutor \emph{Raúl Marticorena} se repite, y solo contamos los proyectos aceptados.
			
			\item Como podemos observar en la imagen \ref{fig:P01Activos} la información estadística es correcta y la tabla muestra bien todos los proyectos activos.
		\end{enumerate}
	
	\item \textbf{Histórico de proyectos}: en esta pantalla debemos observar la información sobre los proyectos históricos. Obtendremos los siguientes datos:
			\begin{enumerate}
			\item Total de 10 proyectos y 10 alumnos asignados.
			\item Calificación media de 8.2
			\item Tiempo medio de realización: 275,6 días.
			\item 6 proyectos son del curso 2013/2014 y 4 de ellos del curso 2014/2015. 
			\item El curso 2013/2014 tiene 9 tutores asignados y el curso 2014/2015 4 profesores en total.
			
			\item En la imagen \ref{fig:P01HistProyectos} la información estadística es correcta y la tabla muestra bien todos los proyectos activos.
		\end{enumerate}
	
	\item \textbf{Histórico del profesorado}: en esta pantalla realizaremos varias pruebas para comprobar que recoge los datos correctamente. Los resultados que debemos obtener son los siguientes:
		\begin{enumerate}
			\item Para el tutor César Ignacio García tenemos dos proyectos ambos durante el curso 2013/2014 y ambos con un tutor 2 perteneciente a la EPS, por lo que cada uno suma 0.5, el resultado que debemos obtener es 1.0 en el curso 2013/2014. Observar imagen \ref{fig:P01GraficaCesar}.
			\item Para el área \emph{Lenguajes y Sistemas informáticos}. Este área tiene tutores que aparecen en la pestaña de históricos como, para el curso 2013/2014 César Ignacio, Raúl Marticorena, Carlos López y José Francisco Díez Pastor, un total de 3 proyectos dirigidos y 2 codirigidos, lo que hace una suma de 4 proyectos en total. Para el curso 2014/2015 tenemos dos tutores que pertenece a este área, Carlos López Nozal y Carlos Pardo Aguilar, que suman dos proyectos dirigidos exclusivamente por ellos. 
			
			\item Para el departamento \emph{Ingeniería Informática}, que contiene el área \emph{Ciencia de la Computación e Inteligencia Artificial} además del mencionado en el punto anterior. Obtendríamos los mismos datos y habría que sumar la participación de tutores como Álvaro Herrero Cosío, María Belén Vaquerizo García y Bruno Baruque que, en el curso 2013/2014 tienen dos participaciones como tutores de un mismo proyecto, sumando 1.0 y en el curso 2014/2015 suman un total de 1.0 por la tutoría de Álvaro en solitario.
			
			\item Los resultados de las últimas dos pruebas deberían ser los siguientes \ref{fig:P01GraficaAD}.
		\end{enumerate}

	
	\item \textbf{Creación de informes}: esta pantalla genera informes con el número de proyectos dirigidos, codirigidos y el número de créditos asignados a un tutor de un área seleccionada. Realizaremos varias pruebas para un total de 10 alumnos:
		\begin{enumerate}
			\item Área de \emph{Lenguajes y Sistemas informáticos}. De este área tenemos varios tutores mencionados en el apartado anterior, y que cuenten con proyectos activos y con alumno asignado, es decir, eliminamos los pendientes. Los resultados deben de ser los siguientes:
				\begin{enumerate}
					\item Raúl Marticorena Sánchez: 1 proyecto dirigido, 0.132 ECTS.
					\item Carlos Pardo Aguilar: 1 proyecto codirigido, 0.132 ECTS.					
					\item José Manuel Sáiz Díez: ningún proyecto activo pero es miembro del tribunal, 0.44 ECTS.
					\item Pedro Latorre Carmona: 1 proyecto codirigido, 0.132 ECTS.						
					\item José Francisco Díez Pastor: 1 proyecto codirigido, 0.132 ECTS.										
					\item Álvar Arnaiz González: ningún proyecto activo pero es miembro del tribunal, 0.44 ECTS.					
				\end{enumerate}
			\item Área de \emph{Ciencia de la Computación e Inteligencia Artificial}. Los resultados deben de ser los siguientes:
			\begin{enumerate}
				\item Ángel Arroyo Puente, ningún proyecto activo pero es miembro del tribunal, 0.44 ECTS.							
			\end{enumerate}
			\item El informe que se genere deberá ser el mismo que se puede observar en la carpeta \texttt{Documentacion/Pruebas} con el nombre \texttt{InformeP01}.
		\end{enumerate}

	
	\item \textbf{Proponer TFG}: esta pantalla permite al usuario administrador proponer proyectos a la base de datos. Se realizarán varias pruebas.
			\begin{enumerate}
				\item Parámetros por defecto: 
					\begin{enumerate}
				 		\item El último proyecto del este archivo tiene como título \emph{GII 21.04 ....}, por lo que el título de este se debería de modificar y al estar en el curso 2022/2023 debería ser 22.01.
				 		\item El tutor debe de ser el usuario que ha iniciado sesión, en este caso figura debería figurar el mio ya que me he dado permisos de administrador, en tu caso debería figurar el tuyo.
				 		\item El alumno1 por defecto es `Aalumnos sin asignar'.
				 		\item La fecha debe de ser la del día en la que se propone el trabajo.
				 		\item Se puede observar en la imagen su funcionamiento \ref{fig:P01SubirTFG}.
					\end{enumerate}
				\item Subida correcta del proyecto: una vez se ha propuesto el trabajo, debemos comprobar que ha sido correctamente generado.
					\begin{enumerate}
						\item En la pantalla de \emph{Administrar TFGs} deberá aparecer el proyecto con estado \emph{Pendiente}.
						\item Si descargamos la base de datos deberá aparecer en la pestaña \texttt{N2 Proyecto}.						
					\end{enumerate}				
			\end{enumerate}

	
	\item \textbf{Aceptar/Denegar TFG}: en esta pantalla deberán aparecer todos los proyectos de la pestaña \texttt{N2 Proyecto} de nuestra base de datos independientemente de su estado. Estado inicial en la imagen \ref{fig:P01Inicial}. Pruebas a realizar:
		\begin{enumerate}
			\item Aceptar TFG: si aceptamos un proyecto deberá cambiar su estado en esta misma pantalla y se sumará en la pestaña de activos. Ver imágenes \ref{fig:P01SelectTFG},  \ref{fig:P01AceptTFG}, \ref{fig:P01PostActivos}
			\item Denegar TFG: si denegamos un proyecto éste será eliminado de la base de datos. Ver imágenes \ref{fig:P01SelectTFG2},
			\ref{fig:P01DenegTFG}.
		\end{enumerate}

	
	\item \textbf{Modificar TFG}: para comprobar que está funcionalidad funciona debemos primero seleccionar un proyecto en la pestaña \emph{Administrar TFGs}. Realizaremos varias pruebas:
		\begin{enumerate}
			\item Modificar datos y mantener activo: primero deberemos \emph{clickar} en el botón \emph{No} en pregunta de la parte inferior de la pantalla, ver imagen \ref{fig:P01No}. A continuación, modificaremos los campos que queramos y pulsaremos en el botón \emph{Aceptar cambios y dejar abierto}. Esta información se podrá ver actualizada en la pestaña de activos si era un proyecto aceptado o en la pestaña de administración de proyectos, también cambiará en la base de datos.
			\item Modificar datos y cerrarlo: primero deberemos \emph{clickar} en el botón \emph{No} en pregunta de la parte inferior de la pantalla, ver imagen \ref{fig:P01No}, esto nos permitirá introducir información como la nota, fecha de presentación o el enlace \emph{URL} del proyecto. A continuación, modificaremos los campos que queramos y pulsaremos en el botón \emph{Aceptar cambios y mover a histórico}. Se verán los cambios de igual forma que en el apartado anterior.			
		\end{enumerate}
\end{enumerate}

\imagenflotante{P01Activos}{Comprobación pantalla activos.}{0.9}
\imagenflotante{P01HistProyectos}{Comprobación pantalla histórico de proyectos.}{0.7}
\imagenflotante{P01GraficaCesar}{Comprobación gráfica de proyectos dirigidos por César Ignacio García.}{0.7}
\imagenflotante{P01GraficaAD}{Comprobación proyectos históricos areas y departamentos.}{0.9}
\imagenflotante{P01SubirTFG}{Comprobación campos de propuesta de proyecto.}{0.9}
\imagenflotante{P01Inicial}{Estado inicial de los proyectos.}{0.9}
\imagenflotante{P01SelectTFG}{Seleccionamos un proyecto para aceptar.}{0.9}
\imagenflotante{P01AceptTFG}{Comprobación cambio de estado.}{0.9}
\imagenflotante{P01PostActivos}{Comprobamos que se ha añadido en activos.}{0.9}
\imagenflotante{P01SelectTFG2}{Seleccionamos un proyecto para denegar.}{0.9}
\imagenflotante{P01DenegTFG}{Comprobamos que se ha eliminado de la base de datos.}{0.9}
\imagenflotante{P01No}{Clickar en la opción 'No'.}{0.9}

\subsection{Archivo de prueba: Prueba02}
Al igual que en el apartado anterior realizaremos pruebas con otro excel que se encuentra en la misma ruta y contiene diferente información. Ahora solo comprobaremos los cambios que se obtienen en las estadísticas generadas por la aplicación.

Tiene un total de 48 proyectos en la pestaña de activos, cinco de ellos con un estado \emph{Pendiente} y un total de 52 en la de históricos.

\begin{enumerate}
	\item \textbf{Proyectos activos}: en esta pantalla se muestran las estadísticas sobre los proyectos de la pestaña \texttt{N2 Proyecto}. Los resultados que debemos esperar son los siguientes:
	\begin{enumerate}
		\item 43 proyectos activos, ya que habiendo un total de 48, 5 de ellos son con estado \emph{Pendiente}, por lo que no se contarán.
		\item 18 proyectos sin asignar.
		\item 6 alumnos sin proyecto asignado, obtenidos en la pestaña \texttt{N2 Alumno}.
		\item 31 tutores asignados.
		
		\item  Como podemos observar en la imagen \ref{fig:P02Activos} la información estadística es correcta y la tabla muestra bien todos los proyectos activos.
	\end{enumerate}
	
	\item \textbf{Histórico de proyectos}: obtendremos los siguientes datos:
	\begin{enumerate}
		\item Total de 52 proyectos y 52 alumnos asignados.
		\item Calificación media de 8.02
		\item Tiempo medio de realización: 284.35 días.		
		\item Tabla con estadísticas \ref{P02Historico1}.
		
		 \begin{table}[]
			\label{P02Historico1}
			\centering
			\begin{tabular}{|l|c|c|c|}
				\hline
				\multicolumn{1}{|c|}{\textbf{}} & \textbf{2014/2015} & \textbf{2015/2016} & \textbf{2018/2019}\\\hline
				Nota media & 7.67 & 8.62  & 7.89 \\ \hline
				Nº de meses & 11.40  & 6.31 & 9.18 \\ \hline
				Nº de proyectos  & 3  & 8 &19 \\ \hline
				Nº de tutores  &  5 &12  & 28  \\ \hline
			\end{tabular}
			\caption{Estadísticas del excel Prueba02. Parte 1}
		\end{table}
	
	\begin{table}[]
		\label{P02Historico2}
		\centering
		\begin{tabular}{|l|c|c|c|c|}
			\hline
			\multicolumn{1}{|c|}{\textbf{}} & \textbf{2019/2020}
			& \textbf{2020/2021}	& \textbf{2021/2022} & \textbf{2022/2023}\\\hline
			Nota media & 7.43 &7.33 & 9.50& 8.20 \\ \hline
			Nº de meses  & 11.33 & 10.18& 6.88	& 9.99 \\ \hline
			Nº de proyectos    &7  &6 & 5	&4  \\ \hline
			Nº de tutores  & 12 &11 & 7	& 8   \\ \hline
		\end{tabular}
		\caption{Estadísticas del excel Prueba02. Parte 2}
	\end{table}
		
	\end{enumerate}
	
	\item \textbf{Histórico del profesorado}:  los resultados que debemos obtener son los siguientes:
	\begin{enumerate}
		\item Tutor \emph{Raúl Marticorena Sánchez}, 2 tfgs codirigidos en el curso 2015/2016, 2 proyectos dirigidos y tres codirigidos durante el curso 2018/2019, pero en uno de ellos el tutor2 no es de la EPS, por lo que cuenta como 1 entero, y uno codirigido, también con un tutor que no es de la EPS, en el curso 2019/2020. Deberíamos obtener entonces 1 en el curso 2015/2016, 4 en el curso 2019/2020 y 1 en el curso 2020/2021. El resultado se puede ver en la imagen \ref{fig:P02GraficaRaul}.
		
		\item Tras realizar los cálculos a mano para el área de  \emph{Lenguajes y Sistemas informáticos}, se obtienen un total de 1 proyecto en 2014/2015, 4 en 2015/2016, 8 en 2018/2019, 4 en 2019/2020, 3 en 2020/2021 y 1 en 2022/2023. Los resultados deberían ser los siguientes \ref{fig:P02GraficaAD}.
	\end{enumerate}	
	
	\item \textbf{Creación de informes}: realizaremos varias pruebas para un total de 30 alumnos:
	\begin{enumerate}
		\item El área de \emph{Lenguajes y Sistemas informáticos} tiene un total de 30 proyectos asignados durante este curso.
		\item El área de \emph{Ciencia de la Computación e Inteligencia Artificial} tiene un total de 6 proyectos asignados durante este curso.
		\item El informe que se genere deberá ser el mismo que se puede observar en la carpeta \texttt{Documentacion/Pruebas} con el nombre \texttt{InformeP02}.
	\end{enumerate}
\end{enumerate}
\imagenflotante{P02Activos}{Comprobación pantalla proyectos activos.}{0.9}

\imagenflotante{P02GraficaRaul}{Comprobación gráfica de proyectos dirigidos por Raúl Marticorena Sánchez.}{0.9}
\imagenflotante{P02GraficaAD}{Comprobación gráfica de los proyectos del área de Lenguajes y Sistemas Informáticos.}{0.9}
\section{Validaciones del sistema}

En esta sección se detallarán cómo la aplicación realiza validaciones antes de
ejecutar las operaciones que el usuario desea en las nuevas pantallas añadidas.

\subsection{Creación de informes}

En esta pantalla se hacen varias comprobaciones antes de la creación del informe como se puede ver en la imagen \ref{fig:validacionInforme}.
En ella se pueden ver los campos obligatorios para la creación del informe, así como una notificación y avisos al usuario.

\imagenflotante{validacionInforme}{Validación campos del informe.}{0.9}

\subsection{Subida de TFGs}

En esta funcionalidad, al igual que en la pantalla de creación de informes, el usuario debe introducir una serie de campos obligatorios si desea proponer un proyecto. Se le avisa al usuario de los campos obligatorios como se puede ver en la imagen \ref{fig:validacionSubirTfg}.

\imagenflotante{validacionSubirTfg}{Validación campos de propuesta de TFG.}{0.9}

\subsection{Administrar TFGs}
A la hora de administrar los \emph{TFGs}, tanto para aceptar y denegar, como para modificarlos se realizan diferentes validaciones.

A la hora de y denegar se debe seleccionar al menos un proyecto como se ve en las imágenes \ref{fig:aceptTFG} y \ref{fig:denegTFG}.

\imagenflotante{aceptTFG}{Validación seleccionar TFG para aceptar.}{0.9}
\imagenflotante{denegTFG}{Validación seleccionar TFG para denegar.}{0.9}

Una vez se ha seleccionado más de un proyecto se vuelve a realizar una confirmación al usuario antes de realizar la operación como se puede ver en las imágenes \ref{fig:validAcept} y \ref{fig:validDeneg}.

\imagenflotante{validAcept}{Validación para aceptar TFGs}{0.9}
\imagenflotante{validDeneg}{Validación para denegar TFGs.}{0.9}

A la hora de seleccionar un proyecto para modificar nos salta un aviso si seleccionamos varios o si no seleccionamos ninguno como se ve en la imagen \ref{fig:validModif}

\imagenflotante{validModif}{Validación para modificar TFGs.}{0.9}

\subsection{Manejo de sesiones}

En esta versión se ha añadido la posibilidad de tener varias sesiones iniciadas a la vez en la aplicación, ya que actualmente se han añadido varios roles para los usuarios, es necesario controlar las sesiones que se generan, de esta forma no se machacan los datos de sesión de otros usuarios. Mediante el uso de variables internas se permite a diferentes usuarios iniciar sesión al mismo tiempo en la \emph{web} como se puede ver en la imagen \ref{fig:variasSesiones}, en la que en un navegador se ha iniciado sesión y en otro no.

\imagenflotante{variasSesiones}{Aplicación web con varias sesiones a la vez.}{0.9}

\subsection{Actualización de ficheros}

Aunque esta funcionalidad ya estuviese incluida en la versión anterior se ha probado que el archivo que se sube a la base de datos y el que se descarga, sin realizar ningún tipo de modificación durante su uso, es el mismo, y tiene los mismos datos e información.