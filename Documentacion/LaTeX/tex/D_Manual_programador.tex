\apendice{Documentación técnica de programación}

\section{Introducción}

\section{Estructura de directorios}

\section{Manual del programador}

A continuación se detallará el proceso de instalación de los programas
necesarios para el desarrollo de la aplicación.

\subsection{Instalación de Java}

Actualmente se sigue ejecutando con la versión de Java 11. A pesar de que se necesitará actualizar cuando migremos a la versión 23 de vaadin.

Para ello se debe descargar la \href{https://www.oracle.com/es/java/technologies/javase/jdk11-archive-downloads.html}{página de descargas de Oracle Java SE 11.0} y descargar la versión de JDK 11, correspondiente con el sistema operativo que se posea y su arquitectura, ya sea de 64 o 32 bits. Ver imagen \ref{fig:Descarga_JDK_11}.

Tras escoger la versión según el SO, se leerán y aceptarán las licencias de uso de Oracle \ref{fig:Descarga_JDK11_Licencia}, y se dará a descargar.

\imagenflotante{Descarga_JDK_11}{Descarga de JDK 11}{0.9}

También se deberá cambiar la variable de entorno de Java del sistema.

\imagenflotante{Descarga_JDK11_Licencia}{Descarga JDK 11 Licencia}{0.9}


\subsection{Instalación de Eclipse}
A continuación se instalará un entorno de desarrollo integrado(IDE) para Java, en este caso se ha utilizado \textbf{Eclipse IDE for Enterprise Java Developers} en la versión 2021-12. 

Para descargar el IDE se accederá a la \href{https://www.eclipse.org/downloads/packages/release/2021-12/r}{página de descargas de Eclipse} y descargar la opción correspondiente a nuestro sistema operativo del \textbf{Eclipse Installer 2021-12 R}. Ver imagen \ref{fig:Descargar_IDE}.

\imagenflotante{Descargar_IDE}{Descargar IDE Eclipse}{0.9}

En el caso de los sistemas operativos Windows se descargará un archivo ejecutable que se deberá ejecutar como administrador. Una vez ejecutado se deberá seleccionar la opción ``\textbf{\textit{Eclipse IDE for Enterprise Java Developers}}'' \ref{fig:Eclipse_Installer}. 

\imagenflotante{Eclipse_Installer}{Seleccionar Eclipse}{0.9}


Por último seleccionaremos el JDK (11) que vayamos a utilizar y la carpeta donde queremos instalar nuestro IDE.

\subsection{Instalación del \textit{plugin de Vaadin} para Eclipse}
Una vez se haya instalado Eclipse, se procederá a añadir el plugin de Vaadin para Eclipse. Esto se realizará mediante el \textbf{Eclipse Marketplace de Eclipse} \ref{fig:Eclipse_marketplace}, el cual se encuentra en la opción de ``\textbf{\textit{Help/Eclipse Marketplace...}}'' de la barra de herramientas.

\imagenflotante{Eclipse_marketplace}{Eclipse marketplace}{0.7}

Una vez en el Eclipse Marketplace, se buscará ``\textbf{Vaadin}'' y se pulsará ``\textbf{Go}''. Tras salir el plugin ``\textbf{\textit{Vaadin Plugin for Eclipse}}'', se dará a ``\textbf{Install}'' y comenzará la instalación del plugin \ref{fig:Plugin_Vaadin}. En la imagen ya se muestra una vez instalado.

\imagenflotante{Plugin_Vaadin}{Plugin Vaadin}{0.9}

\section{Compilación, instalación y ejecución del proyecto}

Se explicará como compilar, instalar y ejecutar el proyecto. En el caso de
la ejecución, se detallará como hacerlo desde un terminal y mediante Eclipse (IDE).

\subsection{Descarga del repositorio}
El código fuente se encuentra en el \href{https://github.com/drg1006/Gestor-TFG-2022.git}{repositorio del proyecto} en GitHub. Para descargarlo se deberá hacer click en ``\textbf{\textit{Code}}'' y copiar la URL que aparece en el apartado de ``\textbf{HTTP}''. Con esta URL deberemos ir al ``\textbf{GitHub Desktop}'' y clonar el repositorio \ref{fig:GitHub_Code}.

\imagenflotante{GitHub_Code}{Copiar URL repositorio}{0.9}

Si se desea tener código en local se deberá descargar el zip ``\textbf{\textit{Download ZIP}}'' en la opción ``\textbf{\textit{Code}}'' anteriormente mencionada. Una vez descargado el zip se descomprimirá y abrirá con Eclipse. 

\subsection{Compilación del proyecto}

Para compilar el proyecto en local desde terminal se usará:
\begin{itemize}
	\item Limpiar las dependencias: \textbf{``mvn clean''}.
	\item Instalar dependencias y compilar: \textbf{``mvn install''}.
	\item Instalar en modo producción (para desplegar): \textbf{``mvn package -Pproduction''}.  
	\item Ejecutar test: \textbf{``mvn test''}.	
\end{itemize}

\subsection{Ejecución del proyecto desde local}

Para la ejecución del proyecto en local desde terminal se usará:
\begin{itemize}
	\item Entrar en la terminal que utilicemos.
	\item Acceder a la carpeta donde tenemos nuestro servidor tomcat instalado y entrar en la carpeta \textbf{/bin}.
	\item Ejecutar nuestro servidor local mediante \textbf{startup} \ref{fig:tomcat_cmd}.  
	\item Entrar en el nuestro navegador en la direccion \textbf{localhost:8080}.
	\item Pulsar en la opción Manage App \ref{fig:Manage_App}.
	\item Iniciamos sesión como manager-gui.(Indicado en el archivo /conf/tomcat-users.xml).
	\item Llegaremos a la pantalla \ref{fig:Desplegar_war} y seleccionaremos el archivo .war que hemos creado al compilar nuestro proyecto con \textbf{``mvn package -Pproduction''}.
	
	\imagenflotante{tomcat_cmd}{Consola con Tomcat ejecutado}{0.9}
	
	\imagenflotante{Manage_App}{Gestor de Aplicaciones de Tomcat}{0.9}
	
	\imagenflotante{Desplegar_war}{Desplegar el archivo .war}{0.9}
	
\end{itemize}

\subsection{Ejecución del proyecto desde Eclipse IDE}

Para la ejecución del proyecto en local desde Eclipse primero debemos importar como proyecto Maven, con el pom.xml. Utilizaremos también un servidor local de de \href{https://tomcat.apache.org/i}{Apache Tomcat}, en concreto, la versión 9. Se puede descargar en \href{https://tomcat.apache.org/download-90.cgi}{la página oficial de Apache Tomcat}.


Una vez descargado y descomprimido, se creará un servicio de Tomcat \ref{fig:añadirTomcat} con la ruta donde se tiene descargado Tomcat y se le dará un nombre \ref{fig:nombreServerTomcat}. Por último, se añadirá el proyecto principal ``sistinf'' \ref{fig:tomcatAñadirProyecto}.

\imagenflotante{añadirTomcat}{Añadir servidor de Tomcat a Eclipse}{0.9}
\imagenflotante{nombreServerTomcat}{Seleccionar carpeta contenedora de Tomcat}{0.9}
\imagenflotante{tomcatAñadirProyecto}{Añadir proyectos a servidor}{0.9}

Para ejecutarlo desde eclipse debemos también seguir todos los pasos de compilación anteriormente mencionados.

Una vez tengamos compilado nuestro código debemos ejecutarlo (click derecho en el proyecto>``Run As''>``Run on Server'').

Si no aparece la vista de los servicios se puede añadir desde la barra de herramientas>``Window''>``Show View''. Para configurar la ruta donde se ejecuta la aplicación, por defecto en \href{http://localhost:8080/}{localhost:8080/} o en ciertos casos \href{http://localhost:8080/sistinf}{localhost:8080/sistinf}.

\subsection{Problemas a la hora de ejecutar el proyecto}

A la hora de ejecutar el proyecto anterior surgieron una serie de problemas tanto para la ejecución por terminal como desde Eclipse.

Cuando quise ejecutarlo mediante la terminal desplegando el archivo .war generado tras compilar me surgía el siguiente error \ref{fig:Error_war}

\imagenflotante{Error_war}{Error tras deplegar el .war en el Gestor de Aplicaciones de Tomcat}{0.9}

Tras buscar información sobre el posible error, se descubre en los logs que proporciona tomcat lo siguiente \ref{fig:error_size}. En el que se informa que se intenta ejecutar un proyecto con un tamaño mayor al que tenemos configurado en tomcat.

\imagenflotante{error_size}{Logs proporcionados por Tomcat}{0.9}

Para solucionar este problema se accede al archivo \emph{apache-tomcat-9.0.68-webapps-manager-WEB-INF} y se modifican las siguientes lineas \ref{fig:size_tomcat} aumentando el número que se indica.

\imagenflotante{size_tomcat}{Logs proporcionados por Tomcat}{0.9}

Cuando quise ejecutarlo mediante Eclipse no me dejaba añadir el proyecto al servidor de \emph{tomcat}, indicando que las versiones no eran compatibles. Por ello se ha entrado en las propiedades del proyecto y se ha cambiado la versión del parametro \emph{Dynamic Web Module} a la 3.1 en el apartado \emph{Project Facets} como se aprecia en \ref{fig:Dynamic} .

\imagenflotante{Dynamic}{Cambio de versión Dynamic Web Module}{0.9}
 
\subsection{Alternativa a Heroku}

A partir del dia 28 de noviembre Heroku dejará de ser gratuito, también ofrecen una alternativa para estudiantes. Esta opción es un acuerdo entre el \href{https://education.github.com}{programa de estudiantes de GitHub} y Heroku. 
Esta colaboración viene explicada en la plataforma de \href{https://blog.heroku.com/github-student-developer-program}{Heroku}.

A pesar de ello este proyecto será útil durante mayor tiempo, y la siguiente persona que quiera continuar con ello no podrá hacerlo permanentemente a través de esta plataforma ya que tras un año será de pago. Aunque su implementación fuese más sencilla se ha decidido cambiar de plataforma.

Por ello mismo migraremos nuestro proyecto a \href{https://northflank.com/}{Northflank}. En su propia página tenemos una serie de \href{https://northflank.com/docs/v1/application/migrate-from-heroku}{guías} para poder migrar nuestro proyecto desde Heroku.

\subsection{Creación aplicación Northflank}

Primero debemos crearnos una cuenta en Northflank como se ve en \ref{fig:creacion_cuenta} y la conectaremos con Github \ref{conectar_Git}.

\imagenflotante{creacion_cuenta}{Crear cuenta Northflank}{0.9}

\imagenflotante{conectar_Git}{Conectarla con GitHub}{0.9}

Por último autorizamos la conexión entre ambas plataformas.

Una vez estemos dentro procederemos a crear nuestro primer proyecto en la pantalla \ref{north_prueba} y le daremos un nombre al proyecto y pulsaremos el botón \emph{"Create project"} \ref{first_project}.

\imagenflotante{north_prueba}{Crear proyecto}{0.9}

\imagenflotante{first_project}{Introducir datos del proyecto nuevo}{0.9}

\subsection{Migración de Heroku a Northflank}

Importaremos desde Northflank el servicio de Heroku que se ha utilizaado hasta ahora.

Primero debemos indicar el link \ref{link_heroku}

\imagenflotante{link_heroku}{Pestaña de importación del proyecto}{0.9}


\section{Pruebas del sistema}
