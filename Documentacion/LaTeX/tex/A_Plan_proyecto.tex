\apendice{Plan de Proyecto Software}

\section{Introducción}
En esta sección se detallará la planificación que se ha realizado, el estudio de viabilidad tanto de la parte económica, como temporal y de la legal. 
\section{Planificación temporal}
Se nombrarán y explicarán brevemente las tareas realizadas a lo largo del proyecto. Estas tareas se encuentran en el \href{https://github.com/drg1006/Gestor-TFG-2022}{repositorio del proyecto en Github}. 

Se añadirán gráficas para una mejor comprensión del tiempo que ha supuesto cada tarea en los (\emph{Sprints}). Esta gráfica muestra el tiempo que se tarda en comenzar y finalizar las tareas de cada uno de los ciclos.

\subsection{Sprint 0 - Puesta a punto (5/10/22 - 19/10/22)}
Puesta a punto del proyecto.
Se procederá a plantear las herramientas con las que se va a trabajar, búsqueda de alternativas y toma de contacto con las herramientas nuevas que se van a emplear.

A continuación se detallarán las tareas que se realizaron durante este primer Sprint:

\begin{itemize}
	
	\item Añadir la extensión ZenHub al navegador. 
	Desde el \textbf{Chrome Web Store} de Google Chrome se añadió la extensión \textbf{ZenHub for GitHub}. 
	
	\item Clonar en repositorio en local. Para clonarlo se ha utilizado la herramienta \textbf{Github Desktop}. Mediante en enlace \href{https://github.com/drg1006/Gestor-TFG-2022.git}{HTTP} que proporciona \emph{Github}.
	
	\item Documentación sobre Vaadin. Se procederá a estudiar el \emph{framework} Vaadin con el que se va a trabajar. A través de la página oficial de \href{https://vaadin.com/}{Vaadin} se realiza la instalación en nuestro entorno IDE \href{https://www.eclipse.org/ide/}{Eclipse} y el aprendizaje.
	
	\item Instalación JDK 11 o superior. Para utilizar la última versión de Vaadin se descargará el  \href{https://www.oracle.com/java/technologies/downloads/#jdk17-windows}{openjdk 17}. 
	
	\item Importación de un proyecto Vaadin de prueba a Eclipse. Para probar el correcto funcionamiento de Vaadin descargaremos e importaremos el proyecto de \href{https://vaadin.com/docs/latest/guide/quick-start}{prueba}.
	
	\item Clonación e imitación del repositorio en Eclipse. Trataremos de clonar e imitar el funcionamiento de la versión \href{https://github.com/dbo1001/Gestor-TFG-2021}{anterior del proyecto} sobre la que trabajamos.
	Posteriormente se descargará también el \href{https://www.oracle.com/java/technologies/downloads/#java11-windows}{openjdk 11} para tratar de clonar el repositorio que estaba en la anterior versión del proyecto. También debemos instalar la herramienta \href{https://tomcat.apache.org/}{Tomcat}.
	
	\item Comienzo de la documentación. Para ello hemos instalado las herramientas TexStudio y MikTex como se indica en \href{https://github.com/ubutfgm/plantillaLatex}{la plantilla Latex} y se ha buscado información para iniciar la documentación.
	
	\item Actualización del README.md. 
	Se modificó el README.md del proyecto para que refleje los cambios respecto a la versión anterior. 
	
	\item Búsqueda de trabajos relacionados con la gestión de TFG/TFM.
	Se realizó una investigación con el fin de encontrar proyectos similares a la aplicación web, es decir, que consistan en la gestión de trabajos de fin de grado o similares. Los proyectos encontrados serán explicados en el apartado \textbf{Trabajos relacionados} de la memoria.
	
	Se puede ver el trascurso de estas tareas en la ilustración \ref{fig:Sprint0_CONTROL}.
	
	\imagenflotante{Sprint0_CONTROL}{Gráfica Burndown Report- Sprint 1}{0.9}
\end{itemize}

\subsection{Sprint 1 - (19/10/22 - 9/11/22)}

Se procederá a estudiar el código del repositorio y a documentar el anexo.

A continuación se detallarán las tareas que se realizaron durante este primer Sprint:

\begin{itemize}
	
	\item Comienzo de la documentación del anexo. Comenzamos en este Sprint a realizar esta documentación desde TexStudio.
	
	\item Estudio del código de todos los paquetes de la carpeta src.Tanto persistence, como util, ui, security y webService.
	
	\item Se procede a buscar el error que salta al intentar ejecutar el código en local.
	
	Se puede ver el trascurso de estas tareas en la ilustración \ref{fig:Sprint1_CONTROL}.
	
	\imagenflotante{Sprint1_CONTROL}{Gráfica Burndown <<Report-- Sprint 1>>}{0.9}
\end{itemize}

\subsection{Sprint 2 - Comienzo de la programación (10/11/22 - 23/11/22)}

En este sprint se comienza a programar y añadir código principalmente arreglando bugs que existían en la versión anterior. También se investiga sobre una alternativa al uso de Heroku que ahora es de pago.

A continuación se detallarán las tareas que se realizaron durante este segundo Sprint:


\begin{itemize}
	\item Eliminación de la distinción entre mayúsculas y minúscula en los filtros. Anteriormente se tenía que introducir el nombre exacto en una columna para que se aplicase bien el filtro, ahora no existe esa distinción.
	
	\item La \emph{URL} del apartado \emph{Documentos} era errónea y se ha sustituido por la correcta.
	
	\item Actualización apartado \emph{información}. Se ha actualizado la información respecto a los tutores y la última versión.
	
	\item Investigar sobre el webscrapping. En un futuro se deberá realizar un webscrapping con la página de investigación de la ubu, por lo que se ha estudiado en qué consiste y posibles implementaciones.
	
	\item Investigación estadística errónea. La información sobre las columnas \emph{Nota, TotalDias y Repositorio} estaba mal implementada en el archivo \texttt{BaseDeDatosTFGTFM.xls} y se ha cambiado a un formato adecuado.
	
	\item Se actualiza la memoria y el anexo correspondiente al anterior Sprint.
	
	\item Elección de una alternativa a Heroku. Heroku pasa a ser de pago el 28 de noviembre de 2022, por lo que se han buscado diferentes alternativas gratuitas como la versión de \href{https://blog.heroku.com/github-student-developer-program}{Heroku para estudiantes}, una colaboración entre Heroku y \href{https://education.github.com/students}{GitHub for Students} o \href{https://northflank.com/}{Northflank}. 
	
	\item Se realizan diferentes pruebas en las plataformas para decidir cual utilizar, y finalmente se optará por usar Heroku for Students, tras desplegar el proyecto en Heroku con éxito y que al tratar de importarlo a Northflank nos indica que debemos aportar una tasa.
	
	\item Búsqueda de librerías o APIs para realizar el webscraping en nuestro proyecto. Se analizan algunas librerías cómo \href{https://jsoup.org/}{Jsoup}, \href{https://htmlunit.sourceforge.io/}{HTMLUnit} o \href{https://jaunt-api.com/}{Jaunt} y APIs como \href{https://www.octoparse.es/blog/web-scraping-api-para-extraccion-de-datos}{Octoparse}.
	
	\item Pruebas de webscraping en un entorno local para determinar cual utilizar. Se realizan una serie de pruebas(que encontramos en el apartado de Pruebas de la Documentación) y finalmente se consigue obtener el resultado que queremos mediante JSoup, por lo que será nuestra elección.
	
	Se puede ver el trascurso de estas tareas en la ilustración \ref{fig:Sprint2_CONTROL}.
	
	\imagenflotante{Sprint2_CONTROL}{Gráfica Burndown <<Report-- Sprint 2>>}{0.9}
\end{itemize}

\subsection{Sprint 3 - Implementación nuevas pantallas (23/11/22 - 14/12/22)}
En este sprint se implementará  al proyecto el proceso de \emph{webscraping} llevado a cabo en el sprint anterior, también se crearán dos nuevas pantallas.

A continuación se detallarán las tareas que se realizaron durante este segundo Sprint:

\begin{itemize}
	
	\item Guardar los datos del \emph{webscraping} en un archivo csv/xls. Se programa un código que permite guardar los resultados sacados mediante el \emph{webscraping} a los ficheros \texttt{BaseDeDatosTFGTFM.xls} y \texttt{N4 Profesores.csv}.
	
	\item Creación del \emph{mock-up} pantalla de creación de informe. Mediante \href{https://pencil.evolus.vn/}{Pencil} crearemos una vista inicial de lo que queremos que sea nuestra pantalla. Esta pantalla se utilizará para crear un informe de un determinado área a elegir por el usuario y guardará los datos en un archivo \emph{.xls}.
	
	\item Creación del \emph{mock-up} pantalla de estadísticas del profesorad. Creada también mediante \href{https://pencil.evolus.vn/}{Pencil}. Esta pantalla se utilizará para visualizar los históricos de los profesores que queramos de la \emph{EPS}, dependiendo de los departamentos y áreas que se indiquen.
	
	\item Implementación \emph{webscrap} en nuestro proyecto. Se ha introducido esta función en nuestro proyecto.
	
	\item Implementación de la pantalla Generar Informes. Se comienza y se termina de programar esta nueva pantalla. No se han creado más \emph{issues} si no que se iba comentando en esta \emph{issue} los problemas y el continuo desarrollo de la pantalla, así como las dudas planteadas.
	
	\item Implementación de la pantalla Estadísticas del Profesorado. Se comienza y se termina de programar esta nueva pantalla.
	
	\item Corrección de la memoria y anexos. Se corrige los fallos expuestos tras el \emph{feedback} del tutor Álvar Arnaiz de los ficheros latex de memoria y anexo.
	
	Se puede ver el trascurso de estas tareas en la ilustración \ref{fig:Sprint3_CONTROL}.
	
	\imagenflotante{Sprint3_CONTROL}{Gráfica Burndown <<Report-- Sprint 3>>}{0.9}
	
\end{itemize}

\subsection{Sprint 4 - Asignación TFGs (15/12/22 - 11/01/23) }

En este sprint se realizarán varias correcciones de las funcionalidades añadidas previamente y se añadirán dos nuevas pantallas, una para subir TFGs y otra para aceptar o denegarlos.

A continuación se detallarán las tareas que se realizaron durante este segundo Sprint:

\begin{itemize}
	\item Crear la pantalla para proponer TFGs siendo tutor. Se comienza y se termina de programar esta funcionalidad en este sprint tras realizar también las correcciones indicadas por los tutores en las reuniones.
	\begin{itemize}
		\item Creación del formulario.
		\item Modificación del excel.
		\item Indicar parámetros obligatorios antes de crear el TFG.
		\item Poner por defecto varios de los datos a introducir.
	\end{itemize}
	\item Corrección de los aspectos señalados por los tutores en las pestañas añadidas previamente.
	\item Documentación de memoria y anexos. Se sigue llevando a cabo la documentación de ambos informes.
	\item Creación de la pantalla de aceptación de TFGs. Esta funcionalidad permite aceptar o denegar TFGs siendo tutor.
	\begin{itemize}
		\item Crear la tabla con los proyectos con estado \emph{Pendiente}.
		\item Crear una columna con el \emph{CheckBox} para seleccionar los TFGs.
		\item Permitir cambiar el estado mediante los botones \emph{Aceptar o Denegar}.
	\end{itemize}
	\item Indicar de forma visual en que pantalla estamos, modificando el color del \emph{botón} en la barra de navegación.
	\item Actualización del despliegue de la aplicación en \emph{Heroku}, cambiando el plan del proyecto.

	Se puede ver el trascurso de estas tareas en la ilustración \ref{fig:Sprint4_CONTROL}.

	\imagenflotante{Sprint4_CONTROL}{Gráfica Burndown <<Report-- Sprint 4>>}{0.9}
\end{itemize} 

\subsection{Sprint 5 - Modificación TFGs (12/01/23 - 28/01/23) }

En este sprint se realizarán varias correcciones de las funcionalidades añadidas previamente y además se modificará la pantalla de aceptación de TFGs que pasará a ser de administración, pudiendo modificar los proyectos activos.

A continuación se detallarán las tareas que se realizaron durante este segundo Sprint:

\begin{itemize}
	\item Actualización de métodos para obtener los datos de ficheros CSV. Se han implementado los métodos para obtener la información de los ficheros CSV.
	\item Implementación del nuevo sistema de \emph{Login}. Pidiendo un registro de forma opcional para asignar un rol al usuario que inicia sesión y mostrarle así las pantallas a las que puede acceder. Se ha restringido la vista a las distintas pantallas de la aplicación dependiendo de los permisos del usuario que inicia sesión.
	\item Ajustar la nueva barra de navegación ante el usuario que inicia sesión, así como del \emph{sub menú} que se crea en la pantalla de histórico tanto del profesorado como de los proyectos. Se han estudiado también otras formas para realizar la barra de navegación que finalmente no se han implementado por no ser óptimas.
	\begin{itemize}
		\item Realizar un desplegable mediante el componente \emph{MenuBar} de Vaadin, que se descartó porque no podía redimensionar.
		\item Mediante el componente \emph{Tab}. Una barra de navegación que finalmente se descartó ya que era mucho más trabajo del pensado crearla y adaptarla al proyecto.
	\end{itemize}		
	\item Creación de la cuenta de \emph{SonarCloud} para analizar la calidad del código creado y de otros proyectos añadidos, para la pantalla de \emph{Métricas}.
	\item Modificación de la pantalla de aceptación de TFGs que pasa a llamarse \emph{Administrar TFGs}. Se han añadido las siguientes funcionalidades:
	\begin{itemize}
		\item Nuevo botón de \emph{Modificación} de TFGs. Botón que solo permite seleccionar un solo TFG de la tabla.
		\item Creación de la nueva pantalla de modificación de los TFGs. Esta pantalla tiene todos los campos posibles que puede tener un proyecto. Permite modificar los datos del TFG seleccionado para mantenerlo abierto y en la pantalla de activos, o para actualizarlo y cerrarlo moviéndolo a la pestaña de \emph{Históricos}.
		\item Actualización de los datos del TFG seleccionado en el archivo de la base de datos, ya sea eliminándolo de la pestaña de activos y añadiéndolo a la de históricos como actualizándolo unicamente en la pestaña de activos.
	\end{itemize}
	\item Correcciones sobre la memoria y los anexos, casos de uso y gestión de roles.
	\item Comentar código de los ficheros creados, tanto de los métodos cómo comentarios de ayuda durante las ejecuciones.
	\item Corrección de bugs en formatos de variables y de código.
	\item Modificación del campo \emph{CursoAsignacion} que a partir de ahora será designado como \emph{FechaAsignacion}, teniendo que cambiar los valores de la base de datos. Este campo será recogido como \emph{LocalDate} y no como \emph{String} que estaba antes.
	\item Se ha añadido la posibilidad de descargar la base de datos con todos los proyectos, exclusivamente para los administradores.
	\item Se ha actualizado la licencia del proyecto.
	\item Se ha prescindido del campo \emph{Titulo} de la pestaña de históricos, que hacía referencia al \emph{hiperenlace} del proyecto en  \emph{GitHub}.
	\item Comprobación de los campos introducidos a la hora de crear un informe. 
	
	Se puede ver el trascurso de estas tareas en la ilustración \ref{fig:Sprint5_CONTROL}.
	
	\imagenflotante{Sprint5_CONTROL}{Gráfica Burndown <<Report-- Sprint 5>>}{0.9}
\end{itemize}

\subsection{Sprint 6 - Manejo de Sesiones/ Casos de Prueba/ Arreglo Bugs (29/01/23 - 08/01/23) }

En este sprint se realizarán los casos de prueba para comprobar que el funcionamiento de la aplicación es correcto y se realizará el manejo de diferentes sesiones que se crean al acceder a la aplicación, ya que anteriormente se utilizaban variables globales

A continuación se detallarán las tareas que se realizaron durante este segundo Sprint:
\begin{itemize}
	\item Se ha conseguido crear y manejar varias sesiones para la aplicación  al mismo tiempo.
	\item Se han realizado casos de prueba y validaciones nuevas para comprobar el correcto funcionamiento de la aplicación al enfrentarse a nuevos archivos.
	\item Se ha arreglado la pantalla de histórico de proyectos que no recogía bien la información de la base de datos en la versión anterior.
	\item Se ha terminado de arreglar los \emph{bugs} encontrados al crear un informe.
	\item Se ha mejorado la eficiencia del código en ciertas pestañas, como en la obtención del número de \emph{TFGs} por año de la pestaña de histórico de profesorado, aumentando la velocidad de creación de gráficas.
	\item Se ha eliminado código innecesario y se ha comentado gran parte de el para aclarar métodos.
	\item Se han modificado los mensajes de texto que le aparecían al usuario para que fuesen más informativos y concisos.
	\item Se han realizado nuevas validaciones a la hora de intentar ejecutar diferentes operaciones dentro de la aplicación.	
	\item Se ha terminado de documentar la memoria y los anexos.
	
	Se puede ver el trascurso de estas tareas en la ilustración \ref{fig:Sprint6_CONTROL}.
	
	\imagenflotante{Sprint6_CONTROL}{Gráfica Burndown <<Report-- Sprint 6>>}{0.9}
	
\end{itemize}	

\subsection{Planificación temporal}

En esta sección se mostrará una ideal inicial de cómo iba a ser el reparto de horas inicial y cual fue el real de las diferentes tareas durante el desarrollo del TFG. La tabla \ref{horas} muestra una predicción de la división del proyecto. Ver tabla \ref{horas}.

\begin{table}[]
	\label{horas}
	\centering
	\begin{tabular}{|l|r|r|}
		\hline
		\multicolumn{1}{|c|}{\textbf{Tarea}}     & \textbf{Inicial (h)} & \textbf{REAL (h)} \\ \hline
		Instalación y configuración de software y hardware & 5  & 11   \\ \hline
		Programación                            & 160   & 200        \\ \hline
		Documentación memoria y anexos          & 60    & 70      \\ \hline
		Estudio de las herramientas utilizadas y alternativas & 15  & 20  \\ \hline
		Preparación de la presentación		    & 5    & 8         \\ \hline
		\textbf{Total}		    &   \textbf{245} &  \textbf{308}        \\ \hline
	\end{tabular}
	\caption{Planteamiento de horas iniciales.}
\end{table}


\section{Estudio de viabilidad}
En este apartado se detallan los costes que llevaría realizar este proyecto de forma real. Se considerarán los costes de recursos humanos, el material empleado y el \emph{Software} usado. 

\subsection{Viabilidad económica}
Primero se detallarán los costes de los recursos utilizados para la realización de este proyecto.

\begin{itemize}
	\item \textbf{Coste de personal}.
	
	En este apartado se estima el gasto que supone mantener a un empleado. Se calcula que se han realizado unas \emph{300} horas de trabajo durante 4 meses. 
	Con el \textit{salario medio español de un programador} 14,43 €/hora ~\cite{SalarioProgramador} y estimando unas 20 horas semanales obtenemos un salario bruto de:
	

$$ 20\frac{horas}{semana}\times14,43\frac{\text{€}}{hora}\times4\frac{semanas}{mes}=1154\text{€}\hspace{0.5em}al\hspace{0.5em}mes  $$

Este calculo hace referencia al salario bruto del empleado, para calcular el salario real del empleado debemos  obtener los impuestos que la empresa debe pagar por él. Esto se puede consultar en la página oficial de la seguridad social: \url{http://www.seg-social.es/wps/portal/wss/internet/Trabajadores/CotizacionRecaudacionTrabajadores/36537}.

Los impuestos son:
\begin{itemize}
	\tightlist
	\item 23.6\% contingencias.
	\item 5.5\% desempleo.
	\item 0.20\% FOGASA.
	\item 0.60\% formación profesional.
\end{itemize}

\imagenflotante{regimen1}{Régimen general de la Seguridad Social parte 1}{0.9}

\imagenflotante{regimen2}{Régimen general de la Seguridad Social parte 2}{0.9}

Teniendo en cuenta estos impuestos, calculamos el gasto que supone el empleado:

$$\frac{1154\frac{\text{€}}{mes}}{1-(0.236+0.055+0.002+0.006)}=1646,21\text{€}\hspace{0.5em}al\hspace{0.5em}mes$$

Tras realizar todos los cálculos,como el proyecto ha durado alrededor de cuatro meses, el coste total será de 6584,84€.

\item \textbf{Hardware}.
El recurso hardware utilizado es un Lenovo Legion Y540 Coste aproximado: 1200€, que ya ha sido amortizado previamente, por lo que el coste real ha sido de 0 €.
 
\item \textbf{Software}.
Todas las herramientas software utilizadas han sido gratuitas a excepción del despliegue en la nube mediante \emph{Heroku}. Los créditos obtenidos sí que han sido gratuitos gracias a la conexión entre \emph{Github estudiantes y Heroku}, pero realmente supone un gasto de 7 dólares al mes.
La duración del proyecto desplegado ha sido de tres meses por lo que el coste total ha sido de 21 dólares.

 
\item \textbf{Total}.
Desestimando gastos indirectos como el internet al vivir en un piso compartido, los gastos totales son los siguientes \ref{Costes totales}.

\begin{table}[]
	\label{Costes totales}
	\centering
	\begin{tabular}{|l|l|}
		\hline
		\multicolumn{1}{|c|}{\textbf{Tipo de costes}}     & \textbf{Total} \\
		Personal &  6584,84\\ 
		Hardware & 0 € \\
		Software & 21 € \\	\hline
		Total & 6605,84 € \\\hline
	\end{tabular}
	\caption{Costes totales.}
\end{table}

\end{itemize}

\subsection{Viabilidad legal}
Se detallarán las licencias \emph{Software} de cada dependencia que se ha utilizado en el proyecto. En el proyecto se ha usado la misma licencia MIT que se utilizaba en la versión anterior del proyecto que permite la libre distribución del \emph{software}. Ver tabla \ref{Dependencias del proyecto}

\begin{table}[]
	\label{Dependencias del proyecto}
	\centering
	\begin{tabular}{|l|l|}
		\hline
		\multicolumn{1}{|c|}{\textbf{Software}}     & \textbf{Licencia} \\ \hline
		Vaadin & Apache License 2.0 \\ \hline
		Spring Boot Maven Plugin & Apache License 2.0 \\ \hline
		JUnit & Eclipse Public License 1.0 \\	\hline
		CsvJdbc & LGPLv2 \\ \hline
		Codoid Fillo & Apache License, Version 2.0 \\ \hline
	\end{tabular}
	\caption{Licencias de las herramientas Software.}
\end{table}

También existe una cuestión de legalidad a la hora de hacer \emph{webscraping}, ya que no siempre es legal realizar este tipo de acciones sobre algunas páginas web, sobretodo si no tenemos los permisos necesarios. En nuestro caso no aplica ya que lo realizamos sobre una \emph{web} interna, por lo que no existe ningún conflicto a la hora de obtener la información.

