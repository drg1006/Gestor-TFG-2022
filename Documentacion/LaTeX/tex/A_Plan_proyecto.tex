\apendice{Plan de Proyecto Software}

\section{Introducción}
En esta sección se detallará la planificación que se ha realizado, el estudio de viabilidad tanto de la parte económica, como temporal y de la legal. 
\section{Planificación temporal}
Se nombrarán y explicarán brevemente las tareas realizadas a lo largo del proyecto. Estas tareas se encuentran en el \href{https://github.com/drg1006/Gestor-TFG-2022}{repositorio del proyecto en Github}. 

Se añadirán gráficas para una mejor comprensión del tiempo que ha supuesto cada tarea en los ciclos (\emph{Sprints}). 

\subsection{Sprint 0 - Puesta a punto (5/10/22 - 19/10/22)}
Puesta a punto del proyecto.
Se procederá a plantear las herramientas con las que se va a trabajar, búsqueda de alternativas y toma de contacto con las herramientas nuevas que se van a emplear.

A continuación se detallarán las tareas que se realizaron durante este primer Sprint:

\begin{itemize}
	
	\item Añadir la extensión ZenHub al navegador. 
	Desde el \textbf{Chrome Web Store} de Google Chrome se añadió la extensión \textbf{ZenHub for GitHub}.
	
	\item Clonar en repositorio en local. Para clonarlo se ha utilizado la herramienta \textbf{Github Desktop}. Mediante en enlace \href{https://github.com/drg1006/Gestor-TFG-2022.git}{HTTP} que proporciona \emph{Github}.
	
	\item Documentación sobre Vaadin. Se procederá a estudiar el \emph{framework} Vaadin con el que se va a trabajar. A través de la página oficial de \href{https://vaadin.com/}{Vaadin} se realiza la instalación en nuestro entorno IDE \href{https://www.eclipse.org/ide/}{Eclipse} y el aprendizaje.
	
	\item Instalación JDK 11 o superior. Para utilizar la última versión de Vaadin se descargará el  \href{https://www.oracle.com/java/technologies/downloads/#jdk17-windows}{openjdk 17}. 
	
	\item Importación de un proyecto Vaadin de prueba a Eclipse. Para probar el correcto funcionamiento de Vaadin descargaremos e importaremos el proyecto de \href{https://vaadin.com/docs/latest/guide/quick-start}{prueba}.
	
	\item Clonación e imitación del repositorio en Eclipse. Trataremos de clonar e imitar el funcionamiento de la versión \href{https://github.com/dbo1001/Gestor-TFG-2021}{anterior del proyecto} sobre la que trabajamos.
	Posteriormente se descargará también el \href{https://www.oracle.com/java/technologies/downloads/#java11-windows}{openjdk 11} para tratar de clonar el repositorio que estaba en la anterior versión del proyecto. También debemos instalar la herramienta \href{https://tomcat.apache.org/}{Tomcat}.
	
	\item Comienzo de la documentación. Para ello hemos instalado las herramientas TexStudio y MikTex como se indica en \href{https://github.com/ubutfgm/plantillaLatex}{plantillaLatex} y se ha buscado información para iniciar la documentación.
	
	\item Actualización del README.md. 
	Se modificó el README.md del proyecto para que refleje los cambios respecto a la versión anterior. 
	
	\item Búsqueda de trabajos relacionados con la gestión de TFG/TFM.
	Se realizó una investigación con el fin de encontrar proyectos similares a la aplicación web, es decir, que consistan en la gestión de trabajos de fin de grado o similares. Los proyectos encontrados serán explicados en el apartado \textbf{Trabajos relacionados} de la memoria.
\end{itemize}

\section{Planificación temporal}

\section{Estudio de viabilidad}

\subsection{Viabilidad económica}

\subsection{Viabilidad legal}


