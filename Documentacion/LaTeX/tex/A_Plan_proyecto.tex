\apendice{Plan de Proyecto Software}

\section{Introducción}
En esta sección se detallará la planificación que se ha realizado, el estudio de viabilidad tanto de la parte económica, como temporal y de la legal. 
\section{Planificación temporal}
Se nombrarán y explicarán brevemente las tareas realizadas a lo largo del proyecto. Estas tareas se encuentran en el \href{https://github.com/drg1006/Gestor-TFG-2022}{repositorio del proyecto en Github}. 

Se añadirán gráficas para una mejor comprensión del tiempo que ha supuesto cada tarea en los (\emph{Sprints}). Esta gráfica muestra el tiempo que se tarda en comenzar y finalizar las tareas de cada uno de los ciclos.

\subsection{Sprint 0 - Puesta a punto (5/10/22 - 19/10/22)}
Puesta a punto del proyecto.
Se procederá a plantear las herramientas con las que se va a trabajar, búsqueda de alternativas y toma de contacto con las herramientas nuevas que se van a emplear.

A continuación se detallarán las tareas que se realizaron durante este primer Sprint:

\begin{itemize}
	
	\item Añadir la extensión ZenHub al navegador. 
	Desde el \textbf{Chrome Web Store} de Google Chrome se añadió la extensión \textbf{ZenHub for GitHub}.
	
	\item Clonar en repositorio en local. Para clonarlo se ha utilizado la herramienta \textbf{Github Desktop}. Mediante en enlace \href{https://github.com/drg1006/Gestor-TFG-2022.git}{HTTP} que proporciona \emph{Github}.
	
	\item Documentación sobre Vaadin. Se procederá a estudiar el \emph{framework} Vaadin con el que se va a trabajar. A través de la página oficial de \href{https://vaadin.com/}{Vaadin} se realiza la instalación en nuestro entorno IDE \href{https://www.eclipse.org/ide/}{Eclipse} y el aprendizaje.
	
	\item Instalación JDK 11 o superior. Para utilizar la última versión de Vaadin se descargará el  \href{https://www.oracle.com/java/technologies/downloads/#jdk17-windows}{openjdk 17}. 
	
	\item Importación de un proyecto Vaadin de prueba a Eclipse. Para probar el correcto funcionamiento de Vaadin descargaremos e importaremos el proyecto de \href{https://vaadin.com/docs/latest/guide/quick-start}{prueba}.
	
	\item Clonación e imitación del repositorio en Eclipse. Trataremos de clonar e imitar el funcionamiento de la versión \href{https://github.com/dbo1001/Gestor-TFG-2021}{anterior del proyecto} sobre la que trabajamos.
	Posteriormente se descargará también el \href{https://www.oracle.com/java/technologies/downloads/#java11-windows}{openjdk 11} para tratar de clonar el repositorio que estaba en la anterior versión del proyecto. También debemos instalar la herramienta \href{https://tomcat.apache.org/}{Tomcat}.
	
	\item Comienzo de la documentación. Para ello hemos instalado las herramientas TexStudio y MikTex como se indica en \href{https://github.com/ubutfgm/plantillaLatex}{plantillaLatex} y se ha buscado información para iniciar la documentación.
	
	\item Actualización del README.md. 
	Se modificó el README.md del proyecto para que refleje los cambios respecto a la versión anterior. 
	
	\item Búsqueda de trabajos relacionados con la gestión de TFG/TFM.
	Se realizó una investigación con el fin de encontrar proyectos similares a la aplicación web, es decir, que consistan en la gestión de trabajos de fin de grado o similares. Los proyectos encontrados serán explicados en el apartado \textbf{Trabajos relacionados} de la memoria.
\end{itemize}

\subsection{Sprint 1 - (19/10/22 - 9/11/22)}

Se procederá a estudiar el código del repositorio y a documentar el anexo.

A continuación se detallarán las tareas que se realizaron durante este primer Sprint:

\begin{itemize}
	
	\item Comienzo de la documentación del anexo. Comenzamos en este Sprint a realizar esta documentación desde TexStudio.
	
	\item Estudio del código de todos los paquetes de la carpeta src.Tanto persistence, como util, ui, security y webService.
	
	\item Se procede a buscar el error que salta al intentar ejecutar el código en local.
	
	Se puede ver el trascurso de estas tareas en la ilustración \ref{fig:Sprint1_CONTROL}.
	
	\imagenflotante{Sprint1_CONTROL}{Gráfica Control chart- Sprint 1}{0.9}
\end{itemize}

\subsection{Sprint 2 - Comienzo de la programación (10/11/22 - 23/11/22)}

En este sprint se comienza a programar y añadir código principalmente arreglando bugs que existían en la versión anterior. También se investiga sobre una alternativa al uso de Heroku que ahora es de pago.

A continuación se detallarán las tareas que se realizaron durante este segundo Sprint:


\begin{itemize}
	\item Eliminación de la distinción entre mayúsculas y minúscula en los filtros. Anteriormente se tenía que introducir el nombre exacto en una columna para que se aplicase bien el filtro, ahora no existe esa distinción.
	
	\item Url del apartado \emph{Documentos} era errónea y se ha sustituido por la correcta.
	
	\item Actualización apartado \emph{información}. Se ha actualizado la información respecto a los tutores y la última versión.
	
	\item Investigar sobre el webscrapping. En un futuro se deberá realizar un webscrapping con la página de investigación de la ubu, por lo que se ha estudiado en qué consiste y posibles implementaciones.
	
	\item Investigación estadística errónea. La información sobre las columnas \emph{Nota, TotalDias y Repositorio} estaba mal implementada en el archivo \texttt{BaseDeDatosTFGTFM.xls} y se ha cambiado a un formato adecuado.
	
	\item Se actualiza la memoria y el anexo correspondiente al anterior Sprint.
	
	\item Elección de una alternativa a Heroku. Heroku pasa a ser de pago el 28 de noviembre de 2022, por lo que se han buscado diferentes alternativas gratuitas como la versión de \href{https://blog.heroku.com/github-student-developer-program}{Heroku para estudiantes}, una colaboración entre Heroku y \href{https://education.github.com/students}{GitHub for Students} o \href{https://northflank.com/}{Northflank}. 
	
	\item Se realizan diferentes pruebas en las plataformas para decidir cual utilizar, y finalmente se optará por usar Heroku for Students, tras desplegar el proyecto en Heroku con éxito y que al tratar de importarlo a Northflank nos indica que debemos aportar una tasa.
	
	\item Búsqueda de librerías o APIs para realizar el webscraping en nuestro proyecto. Se analizan algunas librerías cómo \href{https://jsoup.org/}{Jsoup}, \href{https://htmlunit.sourceforge.io/}{HTMLUnit} o \href{https://jaunt-api.com/}{Jaunt} y APIs como \href{https://www.octoparse.es/blog/web-scraping-api-para-extraccion-de-datos}{Octoparse}.
	
	\item Pruebas de webscraping en un entorno local para determinar cual utilizar. Se realizan una serie de pruebas(que encontramos en el apartado de Pruebas de la Documentación) y finalmente se consigue obtener el resultado que queremos mediante JSoup, por lo que será nuestra elección.
	
	Se puede ver el trascurso de estas tareas en la ilustración \ref{fig:Sprint2_CONTROL}.
	
	\imagenflotante{Sprint2_CONTROL}{Gráfica Control chart- Sprint 2}{0.9}
\end{itemize}

\subsection{Sprint 3 - Implementación nuevas pantallas (23/11/22 - 14/12/22)}
En este sprint se implementará  al proyecto el proceso de \emph{webscraping} llevado a cabo en el sprint anterior, también se crearán dos nuevas pantallas.

A continuación se detallarán las tareas que se realizaron durante este segundo Sprint:

\begin{itemize}
	
	\item Guardar los datos del \emph{webscraping} en un archivo csv/xls. Se programa un código que permite guardar los resultados sacados mediante el \emph{webscraping} a los ficheros \texttt{BaseDeDatosTFGTFM.xls} y \texttt{N4 Profesores.csv}.
	
	\item Creación del \emph{mock-up} pantalla de creación de informe. Mediante \href{https://pencil.evolus.vn/}{Pencil} crearemos una vista inicial de lo que queremos que sea nuestra pantalla. Esta pantalla se utilizará para crear un informe de un determinado área a elegir por el usuario y guardará los datos en un archivo \emph{.xls}.
	
	\item Creación del \emph{mock-up} pantalla de estadísticas del profesorad. Creada también mediante \href{https://pencil.evolus.vn/}{Pencil}. Esta pantalla se utilizará para visualizar los históricos de los profesores que queramos de la \emph{EPS}, dependiendo de los departamentos y áreas que se indiquen.
	
	\item Implementación \emph{webscrap} en nuestro proyecto. Se ha introducido esta función en nuestro proyecto.
	
	\item Implementación de la pantalla Generar Informes. Se comienza y se termina de programar esta nueva pantalla. No se han creado más \emph{issues} si no que se iba comentando en esta \emph{issue} los problemas y el continuo desarrollo de la pantalla, así como las dudas planteadas.
	
	\item Implementación de la pantalla Estadísticas del Profesorado. Se comienza y se termina de programar esta nueva pantalla.
	
	\item Corrección de la memoria y anexos. Se corrige los fallos expuestos tras el \emph{feedback} del tutor Álvar Arnaiz de los ficheros latex de memoria y anexo.
	
	Se puede ver el trascurso de estas tareas en la ilustración \ref{fig:Sprint3_CONTROL}.
	
	\imagenflotante{Sprint3_CONTROL}{Gráfica Control chart- Sprint 3}{0.9}
	
\end{itemize}

\subsection{Sprint 4 - }

\section{Estudio de viabilidad}
En este apartado se detallan los costes que llevaría realizar este proyecto de forma real. Se considerarán los costes de recursos humanos, el material empleado y el \emph{Software} usado. 

\subsection{Viabilidad económica}
Los recursos utilizados para la realización de este proyecto son los siguientes:
\begin{itemize}
	\item  \textbf{Lenovo Legion Y540} Coste aproximado: 1200€.
	\item  \textbf{Eclipse IDE} como entorno de desarrollo del código. Coste: gratuito.
	\item  \textbf{TexStudio} para realizar la documentación. Coste: grautito.
	\item  \textbf{Github Desktop:} como herramienta para actualizar el directorio de github. Coste: grautito.
	\item \textbf{Heroku:} como herramienta de despliegue del proyecto en la nube. Coste: 5 €/mes aproximado.
	\item \textbf{Maven:}. Coste: gratuito.
	\item \textbf{Tomcat:}. Coste: gratuito.
	\item \textbf{Tiempo empleado:} aproximadamente 250 horas, que con el salario medio español de un programador (14,43 €/hora) es 3607,50 €.\textbf{\textit{Fuente}}~\cite{SalarioProgramador}
\end{itemize}

\subsection{Viabilidad legal}
Se detallaran las licencias \emph{Software} de cada dependencia que se ha utilizado en el proyecto.En el proyecto se ha usado la licencia MIT que permite la libre distribución del \emph{software}.

\begin{table}[]
	\label{Dependencias del proyecto}
	\centering
	\begin{tabular}{|l|l|}
		\hline
		\multicolumn{1}{|c|}{\textbf{Software}}     & \textbf{Licencia} \\ \hline
		Vaadin & Apache License 2.0 \\ \hline
		Spring Boot Maven Plugin & Apache License 2.0 \\ \hline
		JUnit & Eclipse Public License 1.0 \\	\hline
		CsvJdbc & LGPLv2 \\ \hline
		Codoid Fillo & Apache License, Version 2.0 \\ \hline
	\end{tabular}
	\caption{Licencias de las herramientas Software.}
\end{table}

También existe una cuestión de legalidad a la hora de hacer \emph{webscraping}, ya que no siempre es legal realizar este tipo de acciones sobre algunas páginas web, sobretodo si no tenemos los permisos necesarios. En nuestro caso no aplica ya que lo realizamos sobre una \emph{web} interna, por lo que no existe ningún conflicto a la hora de obtener la información.

\subsection{Planificación temporal}

En esta sección se mostrará una ideal inicial de como iba a ser el reparto de horas de las diferentes tareas durante el desarrollo del TFG. La tabla \ref{horas} muestra una predicción de la división del proyecto.

\begin{table}[]
	\label{horas}
	\centering
	\begin{tabular}{|l|c|}
		\hline
		\multicolumn{1}{|c|}{\textbf{TAREA}}     & \textbf{TIEMPO} \\ \hline
		Instalación y configuración de software y hardware & 6 h  \\ \hline
		Programación                            & 140 h         \\ \hline
		Documentación memoria y anexos          & 60 h          \\ \hline
		Estudio de las herramientas utilizadas y alternativas & 10 h \\ \hline
			Preparación de la presentación		    & 5 h           \\ \hline
	\end{tabular}
	\caption{Planteamiento de horas iniciales.}
\end{table}
