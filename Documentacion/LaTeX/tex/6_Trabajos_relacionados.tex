\capitulo{6}{Trabajos relacionados}
Se nombrarán algunos de los proyectos y aplicaciones similares o relacionados con la gestión de trabajos de fin de grado o master que se han visitado para la obtención de información del proyecto.

Varios de los TFGs visualizados son los mismos que menciona Diana en la \href{https://github.com/dbo1001/Gestor-TFG-2021}{versión anterior} por lo que no se comentarán.  

\section{GII 20.09 Herramienta web repositorios de TFGII}
Este el proyecto que se propone para mejorar consiste en una aplicación web para la gestión de TFG de la Universidad de Burgos. Este proyecto es una actualización de la versión previa encontrada en \href{https://github.com/jfb0019/Gestor-TFG-2016}{repositorio de GitHub}. 

El código fuente se puede encontrar en este \href{https://github.com/dbo1001/Gestor-TFG-2021}{repositorio de GitHub}. En el anexo se ha explicado como lo hemos compilado y ejecutado para ver su funcionamiento y seguir con el desarrollo. Es un trabajo realizado mediante Java 11, con Vaadin 14 como plataforma de código abierto para la interfaz web. 

\section{Guardians. Development of a web application facilitating communication between teachers and guardians.}

Este proyecto es una aplicación realizada mediante \emph{Vaadin} que podemos observar en el siguiente \href{https://github.com/david-romero/tfg-vaadin}{enlace}.

Es una aplicación que ayuda y simplifica el trabajo diario de los profesores. El profesor puede crear eventos, examinar a sus alumnos, comprobar el progreso de sus alumnos y otras funcionalidades útiles. Las tecnologías que se han utilizado mayoritariamente son \emph{Spring MVC y Vaadin}.

En este proyecto podemos observar el funcionamiento de los roles de los usuarios de una plataforma así como del propio uso de \emph{Vaadin}.

\section{Aplicación web para la gestión de TFG's en una universidad}

Este proyecto tiene como finalidad gestionar los proyectos de una universidad como el nuestro. Podemos ver su funcionamiento en el siguiente \href{https://github.com/amujica/Gestion-TFG}{enlace}.

De este proyecto podemos sacar como está estructurada la aplicación mediante las vistas, y como se organiza mediante DAOS (Organización Autónoma Descentralizadas).
Está desarrollado en java y funciona mediante \emph{servlets}.
Los \emph{servlets} son programas Java™ que utilizan la interfaz de programación de aplicaciones (API) de \emph{servlet} Java. Se ejecutan en un servidor web habilitado para Java, amplían las funciones de un servidor web, de forma similar a como se ejecutan \emph{applets} en un navegador, y amplían las funciones de un navegador ~\cite{servlets}.

