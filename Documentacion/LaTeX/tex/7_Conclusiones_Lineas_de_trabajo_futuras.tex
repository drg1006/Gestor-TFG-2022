\capitulo{7}{Conclusiones y Líneas de trabajo futuras}

\section{Conclusiones}

Una vez finalizado el proyecto y remontando la vista atrás al comienzo del mismo puedo afirmar que se han obtenido muchos conocimientos.

La mayoría de ellos están relacionados con el \emph{framework de Vaadin} y su funcionamiento.
El proyecto ha cumplido con todas las propuestas planteadas al inicio del curso y además se han ido supervisando las diferentes formas de adaptarlas mediante ideas de los tutores y mías propias. 
Por ejemplo, en un primer momento sólo se planteaba la opción de proponer proyectos y posteriormente darle a un administrador la oportunidad de aceptarlos y denegarlos. Pero en las alturas finales del mismo también se decidió darle otra vuelta y permitir al administrador modificar los proyectos que estuviesen activos que quisiese, de esta forma se ha tenido que añadir una funcionalidad extra al proceso.

El proyecto propone mejorar una aplicación ya existente por lo que en gran parte del comienzo implica estudiar y comprender cómo funciona y cómo está estructurada la versión anterior. Gran parte del estudio se ve facilitada gracias a los comentarios que estaban expuestos en el código y que también se han aumentado en esta versión, como ayuda para posteriores mejoras.
Se empleó mucho tiempo en conocer el código y sobretodo en entender las herramientas de uso, entre ellas destacar el uso y funcionamiento del ya mencionado \emph{Vaadin} y del la propia herramienta de documentación \LaTeX{} que nunca había sido utilizada.

Gran parte del funcionamiento se basó en la extracción de información sobre el profesorado de la Escuela Politécnica Superior mediante \emph{webscraping}, cuyos datos se utilizan en la mayoría de las pantallas añadidas. Se estudio como realizar esta técnica, y se intentó mediante diferentes librerías dando finalmente con la solución.

He aprendido mucho sobre el funcionamiento de la aplicación y de los protocolos utilizados para realizar las peticiones entre cliente y servidor. En cuanto a los lenguajes, he reforzado los conocimientos que tenía sobre SQL y sobretodo Java, que se cursan a lo largo del grado. También he aprendido un poco sobre CSS, del cual tenía unos conocimientos muy vagos.

Otro de los aspectos más relevantes que se ha tenido que tratar ha sido la gestión del tipo de usuario que iniciaba sesión en la aplicación, cómo obtener su rol y de esta forma restringir las vistas a las que puede acceder. Se estudiaron numerosas alternativas para lograr esta funcionalidad, como la que se puede observar en al rama \emph{TestBranch} del proyecto y finalmente se implantó la más sencilla y efectiva.

\section{Líneas de trabajo futuras}

Algunas de las posibles líneas de trabajo futuras del programa son:

\begin{itemize}
	\item Incluir la \textbf{internacionalización} de la aplicación, traducirlo a otros idiomas como inglés.
	\item Añadir una funcionalidad que \emph{avisase a los tutores de un TFG sobre su estado}. Es decir, enviar un mensaje al tutor de un proyecto que haya sido modificado por un administrador, ya sea modificando su estado entre aceptado y denegado o cambiando su información.
	\item Como se menciona en la anterior versión y que no ha sido implementado se podría añadir la \textbf{distinción de sesiones empleando la información de
	las Cookies}. Esto permitirá el inicio de sesión de varios usuarios a la
	vez. Con los datos obtenidos de las Cookies se podría obtener la región
	desde la cual se está accediendo y aplicarlo para calcular la fecha de
	la última modificación de los ficheros, según la zona horaria.
	\item Mejorar los aspecto visuales de las nuevas pantallas añadidas, ya que en esta versión se ha tenido más en cuenta la funcionalidad.
	\item Permitir a un alumno solicitar un proyecto de la lista de activos que existan desde la propia aplicación, que no lo tenga que hacer mediante un correo referenciando dicho proyecto. Es decir añadir una columna en el apartado de activos que pueda ser <<Solicitado por>>.
	
\end{itemize}