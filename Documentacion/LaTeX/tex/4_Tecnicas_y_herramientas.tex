\capitulo{4}{Técnicas y herramientas}

Se van a presentar las técnicas metodológicas y las herramientas de desarrollo que se han utilizado para llevar a cabo el proyecto. En su mayoría se he optado por elegir herramientas de las que ya se tenían conocimiento anteriormente como: GitHub, Eclipse y GitHub Desktop.

\section{Código Fuente}

\subsection{Java}
Java~\cite{pagina_java} es un lenguaje de programación orientado a objetos usado en el desarrollo de aplicaciones.

\subsection{Maven}
Maven~\cite{pagina_maven} es una herramienta software con una arquitectura basada en plugins, desarrollada por Apache Software Foundation(ASF), usada para gestionar y construir proyectos en Java. Configura el proyecto, a través de un Project Object Model (POM) en formato XML, mediante dependencias con módulos y componentes externos. Además, incluye tareas como la compilación del código, su empaquetado, descarga e instalación de plugins. Existen plugins para trabajar con otros lenguajes como C/C++ y con el Framework .Net. 

\subsection{Vaadin}
Vaadin~\cite{pagina_vaadin} es una plataforma de código abierto para el desarrollo de aplicaciones web con Java. Permite el uso de lenguajes como HTML, CSS y JavaScript,etc.
Para usar Vaadin se requiere de por lo menos un JDK 8 (\textit{Java Development Kit}), un entorno de Desarrollo Integrado como Eclipse, NetBeans o IntelliJ Idea. 

\section{Eclipse}
Eclipse~\cite{pagina_eclipse} es un IDE (entorno de Desarrollo Integrado) de código abierto, multiplataforma y basado en Java. 

\subsection{LaTeX} es un software libre para la composición de textos con una gran calidad tipográfica. Es empleado en gran medida para la creación de artículos, libros técnicos y tesis.

\section{Gestión del proyecto y control de versiones}
Para el control de versiones se ha optado por utilizar programas y plataformas ya conocidas.

\subsection{GitHub}
GitHub~\cite{pagina_github} es una plataforma de repositorios online colaborativos que permite llevar a cabo la gestión de proyectos y el control de versiones.

\subsection{ZenHub}
ZenHub~\cite{pagina_zenhub} es una plataforma de gestión de proyectos totalmente integrada en GitHub. Organiza las issues en el tablero canvas según su estado: recién creadas, pendientes, en proceso, ya terminadas, etc.

\subsection{GitHub Desktop}
Github Desktop~\cite{pagina_github_desktop} simplifica la tarea de conectar el repositorio GitHub sin necesidad de usar la línea de comandos de Git. A través de este programa se pueden realizar commit y subirlos al GitHub(push), bajar los cambios realizados en el repositorio (pull), etc. 