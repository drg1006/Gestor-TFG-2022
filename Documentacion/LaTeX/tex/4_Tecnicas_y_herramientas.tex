\capitulo{4}{Técnicas y herramientas}
En esta sección se detallarán algunas de las técnicas y herramientas utilizadas durante el desarrollo del trabajo y las alternativas que no se han llegado a usar.

Muchas de las herramientas utilizadas están ya explicadas en la anterior versión del proyecto que podemos encontrar en \href{https://github.com/dbo1001/Gestor-TFG-2021}{GII 20.09 Herramienta web repositorios de TFGII}. Por lo tanto lenguajes de programación como \emph{Java, SQL, XML} no se detallarán a pesar de haber sido utilizados, y herramientas como \emph{GitHub, GitHub Desktop, Eclipse IDE, Tomcat o Maven} tampoco se explicarán.

\section{Herramientas de Desarrollo}

\subsection{Heroku para estudiantes}
Heroku~\cite{heroku_students} es una plataforma basada en la nube como servicio ( PaaS ) para construir, ejecutar y administrar aplicaciones. 
La plataforma Integración de GitHub le permite conectar su aplicación Heroku a su repositorio GitHub e implementar en cada empuje a GitHub.
Heroku ofrece una gama de servicios de bajo costo para ayudarlo a experimentar, aprender y crear prototipos de nuevas ideas. Para los estudiantes de GitHub, vamos un paso más allá y agregamos aún más recursos a su Paquete de desarrolladores de estudiantes de GitHub. 
Heroku pasó a ser de pago el 28 de noviembre de 2022,pero al pertenecer al programa de GitHub para estudiantes proporcionado por la UBU, podemos solicitarlo, y que se nos agreguen unos créditos para poder desplegar nuestro proyecto en la nube,por lo que es la herramienta que hemos utilizado para desplegar nuestro proyecto.

\subsection{Jsoup}
Jsoup~\cite{jsoup} es una biblioteca Java para trabajar con HTML del mundo real. Proporciona una API muy conveniente para recuperar URL y extraer y manipular datos, utilizando los mejores métodos HTML5 DOM y selectores CSS. 
Es la herramienta que hemos utilizado para realizar el \emph{webscraping} ya que es la que mejor se ajustaba y con la que mejor resultado hemos obtenido, también resulta la más sencilla de utilizar.

\subsection{HTMLUnit}
HtmlUnit~\cite{HTMLUNIT} es un "navegador sin GUI para programas Java". Modela documentos HTML y proporciona una API que le permite invocar páginas, completar formularios, hacer clic en enlaces, etc... tal como lo haces en tu navegador "normal. 
Es una de las alternativas planteadas a la hora de utilizar una biblioteca para el \emph{webscrap}.

\subsection{Pencil}
Pencil~\cite{Pencil_project} está construido con el propósito de proporcionar una herramienta de creación de prototipos GUI gratuita y de código abierto que las personas pueden instalar y usar fácilmente para crear maquetas en plataformas de escritorio populares. 
Se ha utilizado para crear los \emph{mock-ups} de las nuevas pantallas que se han desarrollado.

\subsection{Northflank}
Northflank~\cite{Northflank} es su sitio web cabe mencionar que los precios escalan junto a como va desarrollándose tu proyecto. Puedes pagar solo por los recursos que consumen tus servicios durante la compilación y el tiempo de ejecución. Puedes escalar tanto horizontal como verticalmente sin gastos inesperados. Al ser un VPS, puedes trabajar con muchos lenguajes y frameworks. Cuentan con una versión para Developers que es gratuita. 
Finalmente se optó por no utilizar este sistema ya que tras intentar importar nuestro proyecto nos indicaba que era de pago por que superaba el límite de peso permitido para un proyecto gratuito.

\subsection{Vaadin}
Vaadin~\cite{Vaadin} es un framework de desarrollo de SPA que permite escribir el código de dichas aplicaciones en Java o en cualquier otro lenguaje soportado por la JVM 1.6+. Esto permite la programación de la interfaz gráfica en lenguajes como Java 8, Scala o Groovy, por ejemplo.

Uno de las características diferenciadores de Vaadin es que, contrario a las librerías y frameworks de JavaScript típicas, presenta una arquitectura centrada en el servidor, lo que implica que la mayoría de la lógica es ejecutada en los servidores remotos. Del lado del cliente, Vaadin está construido encima de Google Web Toolkit, con el que puede extenderse.


\section{Herramientas de Documentación}

\subsection{Latex}
 Latex~\cite{latex} es un sistema de composición de textos, orientado a la creación de documentos escritos que presenten una alta calidad tipográfica. Por sus características y posibilidades, es usado de forma especialmente intensa en la generación de artículos y libros científicos que incluyen, entre otros elementos, expresiones matemáticas.
 
\subsection{TexStudio}
TexStudio~\cite{TexStudio} es un entorno de escritura integrado para crear documentos LaTeX. Su principal objetivo es hacer que escribir LaTeX sea lo más fácil y cómodo posible. Por esto, TeXstudio nos ofrece a los usuarios numerosas funciones como el resaltado de sintaxis, un visor integrado, la verificación de referencias y varios asistentes, entre otras. 
Es el programa de edición que hemos utilizado para realizar la memoria y los anexos.

\subsection{MikTex}
MiKTeX~\cite{TexStudio} es una implementación de TeX y programas relacionados para Windows. TeX es un sistema de caracteres de alta calidad para la edición de documentos.
Es la herramienta que se utiliza para compilar los documentos que se crean en TexStudio.

\subsection{Overleaf}
Overleaf~\cite{Overleaf}  es una herramienta de publicación y redacción colaborativa en línea que hace que todo el proceso de redacción, edición y publicación de documentos científicos sea mucho más rápido y sencillo. Overleaf brinda la conveniencia de un editor LaTeX fácil de usar con colaboración en tiempo real y la salida totalmente compilada producida automáticamente en segundo plano a medida que escribe.
Era la herramienta que se iba a utilizar en un principio pero se deshechó la idea ya que se vió que era más intuitivo de utilizar la aplicación de escritorio TexStudio.

